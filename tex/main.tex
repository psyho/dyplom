\documentclass[12pt]{report}

%-------------------------------------------------------------------------------
% PAKIETY
%-------------------------------------------------------------------------------

% obsluga jezyka polskiego
\usepackage[utf8x]{inputenc}
\usepackage[OT4]{fontenc}
\usepackage{polski}

% dzielenie wyrazow
\usepackage{hyphenat}

% dla /todo
\usepackage{todonotes}

% skorowidz
\usepackage{makeidx}

\usepackage[dvips, bookmarks, colorlinks=false, pdfauthor={Adam Pohorecki}, pdfsubject={Praca Magisterska}, pdfkeywords={master thesis, nosql, scalability}]{hyperref}

% page headers
\usepackage{fancyhdr}
\setlength{\headheight}{15.2pt}

\pagestyle{fancyplain}
\renewcommand{\chaptermark}[1]{\markboth{#1}{}}
 
\lhead{\fancyplain{}{\textit{\rightmark}}}
\chead{}
\rhead{}
\lfoot{}
\cfoot{\fancyplain{}{\thepage}}
\rfoot{}

\makeindex

\bibliographystyle{apalike}

%-------------------------------------------------------------------------------
% Dane pracy
%-------------------------------------------------------------------------------

\author{Adam Pohorecki}
\title{Optymalizacja wyboru środowiska realizacji baz danych dla potrzeb wysokowydajnych systemów informacyjnych}

% ograniczenia co do tego co ma byc zalaczone w  wygenerowanym pdf
%\includeonly{chapters/definicja_problemu/definicja_problemu}
%\includeonly{chapters/studium_literatury/studium_literatury}
%\includeonly{chapters/opis_rozwiazan/opis_rozwiazan}
%\includeonly{chapters/porownanie_rozwiazan/porownanie_rozwiazan}
%\includeonly{chapters/wnioski/wnioski}

%-------------------------------------------------------------------------------
% Treść pracy
%-------------------------------------------------------------------------------

\begin{document}

% slowa, ktore nie powinny byc dzielone na koncu linii
\hyphenation{
Amazon
BigTable
Cassandra 
Facebook
Google
HBase
MySQL 
NoSQL 
SimpleDB
Twitter 
}

% moje komendy
\newcommand{\id}[1]{\index{#1}}  
\newcommand{\wi}[1]{#1\index{#1}}  
\newcommand{\wwi}[1]{\emph{#1}\index{#1}}  
\newcommand{\mwi}[1]{\textbf{#1}\index{#1}}
\newcommand{\ii}[1]{\textit{#1}}
\newcommand{\myfigure}[3]{\begin{figure}[!ht]\centering\includegraphics[width=0.75\textwidth]{#1}\caption{#2}\label{#3}\end{figure}}


%-------------------------------------------------------------------------------
% Część początkowa
%-------------------------------------------------------------------------------

\maketitle

\tableofcontents

%-------------------------------------------------------------------------------
% Główna treść pracy
%-------------------------------------------------------------------------------

% rozdzialy pracy
\chapter{Wstęp}

Ostatnie lata przyniosły znaczny wzrost popularności nierelacyjnych baz danych.
Nowe systemy odznaczające się wysoką wydajnością i skalowalnością coraz częściej zastępują lub uzupełniają relacyjne bazy danych w systemach produkcyjnych.

Zyskujący na popularności ruch NoSQL (ang. \emph{Not only SQL}) postuluje aby osoby podejmujące decyzje o architekturze systemów informatycznych nie wybierały zawsze relacyjnych baz danych jako mechanizmu persystencji, tylko świadomie dokonywały analizy wad i zalet poszczególnych rozwiązań na potrzeby ich zastosowania.
Ze względu na bardzo dużą liczbę konkurujących ze sobą produktów, bardzo trudne jest wybranie systemu, który najlepiej spełni stawiane przed nim wymagania.

Niniejsza praca ma na celu przybliżenie najpopularniejszych obecnie nierelacyjnych baz danych, ułatwienie zrozumienia stosowanych w tych systemach technik oraz sformułowanie wad, zalet i charakterystyk zastosowań opisanych rozwiązań.

\section{Struktura pracy}

Niniejsza praca składa się z czterech części:

\begin{description}
 \item[Definicja Problemu]
 W rozdziale tym opisane zostały: tradycyjne podejście do skalowania aplikacji internetowych w oparciu o relacyjną bazę danych, problemy z którymi mierzą się systemy NoSQL, oraz historia i znaczenie ruchu NoSQL.
 \item[Studium Problemu]
 W kolejnym rozdziale przedstawione zostały, na podstawie dostępnej literatury, zagadnienia, które miały największe znaczenie w rozwoju systemów NoSQL.
 Opisane w tym rozdziale zostały zarówno prawa rządzące systemami rozproszonymi (\emph{Teoria CAP}), wzorce projektowe dotyczące replikacji (\emph{BASE, Eventual Consistency}) jak i funkcjonujące systemy internetowych gigantów takich jak Amazon i Google, które doczekały się dostępnych na licencji Open Source implementacji oraz wpłynęły w mniejszym lub większym stopniu na wiele baz NoSQL.
 \item[Opis dostępnych rozwiązań]
 Na tą część pracy składa się szereg rozdziałów zaczynając od rozdziału ,,Klasyfikacja Rozwiązań'', a kończąc na rozdziale opisującym ,,Bazy Grafowe''.
 Przybliża ona cechy, wady i zalety najpopularniejszych systemów NoSQL.
 Ponadto opisuje ona podział systemów NoSQL ze względu na takie parametry jak model danych czy sposób replikacji.
 \item[Wnioski]
 Ostatnia część pracy zajmuje się porównaniem możliwych zastosowań opisanych systemów i próbuje odpowiedzieć na pytania kiedy lepiej jest zastosować nierelacyjną bazę danych, a kiedy lepiej pozostać przy relacyjnej.
\end{description}

\section{Cel pracy}

Celem niniejszej pracy jest wprowadzenie czytelnika w świat nierelacyjnych, skalowalnych baz danych.
Przedstawione zostały stosowane w tych systemach techniki oraz cechy, które wyróżniają i łączą poszczególne rozwiązania.
Opisane zostały popularne obecnie systemy oraz zalecenia dotyczące wyboru mechanizmu persystencji dla różnych typów aplikacji.
\chapter{Definicja Problemu}

\section*{Streszczenie}
W poniższym rozdziale zostanie zaprezentowany problem, który usiłują rozwiązać systemy opisane w niniejszej pracy.
Przedstawione zostaną sposoby skalowania aplikacji przy zastosowaniu relacyjnych baz danych.
Opisane zostanie także pojęcie NoSQL i jakie problemy próbuje ten ruch rozwiązać.

\section{Wstęp}
Popularyzacja dostępu do internetu przyczyniła się w ostatnich latach do  powstania licznych serwisów, które mogą się pochwalić dziesiątkami, a nawet setkami milionów odwiedzin dziennie. 
Twórcy serwisów takich jak Facebook, Twitter czy Google zostali postawieni przed problemem stworzenia rozproszonej architektury bazodanowej, która zapewni horyzontalną skalowalność przy równoczesnym zachowaniu wysokiej dostępności. 
Ze względu na występujący jeszcze kilka lat temu brak takich rozwiązań na rynku firmy te były zmuszone do implementacji tego typu baz na własne potrzeby. 
Ostatnimi laty jednak sytuacja zaczęła ulegać zmianie: produkty wytworzone na wewnętrzny użytek zostają upubliczniane na licencjach Open Source, dzięki czemu obecnie mamy już dostęp do przynajmniej kilkunastu rozwiązań umożliwiających dostęp do peta-bajtów danych rozproszonych na tysiącach serwerów.

\section{Skalowalność}
Niniejsza praca stawia przed sobą zadanie wprowadzenia czytelnika w świat wysoce skalowalnych systemów bazodanowych.
Aby jednak tego dokonać musimy najpierw przedstawić, czym jest skalowalność w rozumieniu autora i jakiego typu skalowalnością będziemy się zajmować.

Wyobraźmy sobie aplikację internetową, na przykład stronę społecznościową taką jak Nasza Klasa czy Facebook.
Aplikacje takie jak ta początkowo mają tysiące użytkowników, a gdy odniosą sukces, liczba ta może wzrosnąć do dziesiątek czy nawet setek milionów.
Ponadto aplikacje te podlegają ciągłemu rozwojowi przez wiele lat, co bardzo często pociąga za sobą konieczność wprowadzania zmian w schemacie bazy, a także niejednokrotnie dokonywania migracji danych.

System bazodanowy jest skalowalny w rozumieniu tej pracy, jeżeli koszty jego utrzymania wzrastają co najwyżej liniowo wraz ze wzrostem liczby użytkowników aplikacji. 
Warunkiem koniecznym jest także to, aby czas potrzebny na wykonanie operacji na rzecz użytkownika aplikacji (zapisy, odczyty), nie degradował się znacznie wraz ze wzrostem ilości danych, którymi system zarządza.

\section{Tradycyjne rozwiązania}
Dotychczas najczęściej stosowanym rozwiązaniem problemów skalowalności było zastosowanie bazy danych takiej jak MySQL połączonej z Memcached jako przechowywany w pamięci operacyjnej, rozproszony cache \cite{highscalability-mysql-end-of-an-era}.
Opiszmy pokrótce, jak takie skalowanie aplikacji w uproszczeniu wygląda.

\subsection{Replikacja}
Początkowo cała baza danych mieści się na jednej maszynie i jest w stanie poradzić sobie ze wszystkimi operacjami odczytu i zapisu jakich dokonuje aplikacja.
Wraz ze wzrostem liczby użytkowników serwer zaczyna docierać do granic swojej przepustowości. 
Ponieważ aplikacje internetowe wykonują zazwyczaj dużo więcej odczytów niż zapisów, dodajemy stopniowo kilka kolejnych serwerów replikowanych w konfiguracji master-slave, zwiększając w ten sposób liczbę równoczesnych odczytów, jednak nie zwiększając (a nawet zmniejszając) liczby możliwych zapisów.
Alternatywą jest dodawanie kolejnych serwerów w konfiguracji master-master, która jest zazwyczaj trudniejsza do poprawnego skonfigurowania, ale za to w przypadku gdy jeden z serwerów master zawiedzie, inny z łatwością przejmie jego rolę.

\myfigure{chapters/definicja_problemu/master-slave.png}{Replikacja Master-Slave}{fig:master-slave}

Warto jednak zwrócić uwagę, na to jak szybko takie rozwiązanie się degraduje.
Załóżmy, że początkowo serwer \emph{master} jest obciążony w 50\% zapisami.
Oznacza to, że każdy z węzłów \emph{slave} musi także poświęcić 50\% zasobów na radzenie sobie z replikacją zapisów\footnote{Zakładam, że wszystkie serwery są identyczne}, a resztę może poświęcić na przetwarzanie odczytów.
Wraz ze wzrostem obciążenia serwera \emph{master}, serwery \emph{slave} muszą coraz więcej zasobów poświęcać na replikację zapisów, co zmniejsza ogólną przepustowość systemu.
Ponadto problemem może być także to, że niektóre systemy baz danych, tak jak, np. MySQL dokonują replikacji asynchronicznie.
Powoduje to, że po dokonaniu operacji zapisu, odczyt może zwrócić wartość niezgodą z zapisaną.
Wymaga to dodatkowej pracy ze strony programisty \cite{zaitsev-scaling-mysql}.

\subsection{Partycjonowanie}
Kiedy oczywiste staje się, że nasz serwer nie jest w stanie sobie poradzić z obciążeniem wynikającym z zapisów, mamy dwie drogi do wyboru: wymienić maszynę na mocniejszą, albo zacząć przetrzymywać tylko część danych na pojedynczym serwerze.
Pierwsza opcja, nazywana \emph{skalowaniem wzwyż}, ma pewne zalety: nie wymaga zmian w aplikacji i jej nie komplikuje.
Wadą tego rozwiązania jest jednak cena w porównaniu do możliwości.
Cytując za artykułem na blogu \emph{Coding Horror} \cite{codinghorror-scaling-up-vs-out}: za cenę stu tysięcy dolarów możemy albo kupić jeden potężny serwer z 32 procesorami, 512GB RAM i 4TB przestrzeni dyskowej, albo 83 mniejsze serwery posiadające łącznie 332 rdzenie, 664GB RAM i 40,5TB przestrzeni dyskowej.
Oczywiście nie można w ten sposób skalować aplikacji wzwyż w nieskończoność, ale warto pamiętać, że tylko nieliczne strony mogą konkurować rozmiarami z takimi gigantami jak Facebook.
Większość aplikacji nigdy nie wyrośnie ponad rozmiar, w którym ,,superkomputer'' nie wystarcza już do utrzymania ich obciążenia.

\myfigure{chapters/definicja_problemu/hp-proliant-dl785.jpg}{HP ProLiant DL785 G5}{fig:superkomputer}

Z drugiej strony mamy partycjonowanie danych, czyli \emph{skalowanie aplikacji wszerz}.
W tym rozwiązaniu wraz ze wzrostem liczby użytkowników zamiast wymieniać serwer na większy i mocniejszy dodajemy kolejne maszyny.
Nie możemy jednak całości bazy danych przechowywać na każdym z serwerów - problemy z tym związane zostały opisane powyżej.
W związku z tym, dokonuje się podziału danych (partycjonowania) poziomego, pionowego albo obu równocześnie.

Kiedy mowa o pojedynczej bazie danych, horyzontalne partycjonowanie oznacza, że wiersze na podstawie jakiegoś kryterium trafiają do różnych tabel.
Dla przykładu: tabela użytkowników może być podzielona w taki sposób, że użytkownicy z Polski trafiają do tabeli \verb=polish_users=, natomiast użytkownicy z Niemiec do tabeli \verb=german_users=.
Tego rodzaju partycjonowania dokonuje się zazwyczaj w celu zwiększenia wydajności i niektóre bazy danych (w tym MySQL\footnote{http://dev.mysql.com/tech-resources/articles/performance-partitioning.html}) pozwalają na definiowanie partycjonowania w schemacie bazy danych.
W przypadku rozproszonych baz danych, mówimy zamiennie o partycjonowaniu horyzontalnym bądź shardowaniu (ang. \emph{sharding}).
Różni się ono od partycjonowania poziomego dla pojedynczej bazy tym, że dane zamiast trafiać do różnych tabel, trafiają do tej samej tabeli, ale na różne węzły bazy danych.

\myfigure{chapters/definicja_problemu/sharding.png}{Shardowanie}{fig:sharding}

Partycjonowanie pionowe (wertykalne) podobnie ma dwa znaczenia w zależności od kontekstu.
Dla pojedynczej bazy polega ono na podziale tabel na mniejsze (zawierające tylko część kolumn).
Przypomina to normalizację bazy, ale dzielimy także już znormalizowane tabele, np. po to aby aby część kolumn umieścić na innym dysku, albo aby oddzielić częściej odczytywane dane od rzadziej odczytywanych.
W rozproszonym systemie partycjonowanie pionowe oznacza, że nie połączone ze sobą tabele mogą być umieszczone na różnych węzłach bazy danych.
W ten sposób zapis do jednej tabeli nie obciąża dwóch, tylko jeden serwer.
Oczywiście bardzo często nie jest możliwe znalezienie tabel, które nie są za sobą połączone pośrednio lub bezpośrednio.
Dlatego często takie partycjonowanie wymaga zduplikowania części tabel na obu węzłach.

Opisane wyżej formy partycjonowania systemów rozproszonych najczęściej muszą być obsługiwane przez warstwę aplikacji.
Z tego względu praktycznie niemożliwe jest zapewnienie transakcyjności pomiędzy poszczególnymi węzłami.
Kosztowne i trudne są także operacje joinowania pomiędzy serwerami.

\subsection{Memcached}

Podstawową receptą na przyspieszenie aplikacji internetowych jest cacheowanie.
Najpopularniejszym rozwiązaniem stosowanym do tego celu jest obecnie Memcached.
Memcached to \emph{darmowy i open source, rozproszony, wysokowydajny system do przechowywania obiektów w pamięci, ogólnego zastosowania ale przeznaczony do przyspieszania dynamicznych aplikacji internetowych przez zmniejszenie obciążenia bazy danych}\footnote{ang. Free \& open source, high-performance, distributed memory object caching system, generic in nature, but intended for use in speeding up dynamic web applications by alleviating database load. - http://memcached.org/}.
Ponieważ serwery Memcached nie komunikują się między sobą bezpośrednio, możemy zwiększać przepustowość systemu liniowo, poprzez dodawanie kolejnych maszyn.
Stosowanie Memcached w porównaniu do cache bazy danych ma wiele zalet: cache bazy danych jest ograniczony do pojedynczego serwera, czyszczony przy zapisach i ograniczony w tym, co możemy do niego zapisywać.
Dla porównania nad Memcached mamy o wiele większą kontrolę i większą swobodę co do tego co w nim się znajdzie.
Możemy cacheować pojedyncze wiersze tabel bazy danych, całe tabele, wyniki zapytań wymagających skomplikowanych obliczeń, pliki, fragmenty wygenerowanych stron HTML i wiele innych obiektów.

W architekturach wielu wysokowydajnych aplikacjach internetowych Memcached odgrywa centralną rolę.
W artykule \cite{highscalability-mysql-memcached} czytelnik znajdzie interesujący opis strategii wykorzystania Memcached w serwisie Fotolog.
Szczególnie warta uwagi jest złożoność zaprojektowanego tam systemu, oraz to że fragmenty aplikacji wymagające największej wydajności komunikują się nie z bazą danych bezpośrednio, ale z serwerami cache, w których część tabel jest w całości odwzorowana.

\subsection{Problemy z tradycyjnymi rozwiązaniami}

Tradycyjne rozwiązania, oparte na shardowaniu, replikacji i cacheowaniu sprawdziły się i są w ciągłym użyciu w wielu aplikacjach.
Niestety niosą one ze sobą istotne ograniczenia: shardowanie pociąga za sobą wysoki koszt operacji na zbiorach danych, które są rozproszone na wielu węzłach systemu i brak transakcji, asynchroniczna replikacja wymaga aby odczyty i zapisy pojedynczego klienta były kierowane do tego samego serwera a cache wymaga skomplikowanych procedur inwalidacji\footnote{Poprzez ,,inwalidację'' rozumiem oznaczanie części zapisanych w cache danych jako nieaktualne, usunięcie ich lub zastąpienie aktualnymi. W praktyce bardzo trudno jest zapewnić aktualność danych zapisanych w cache w skomplikowanym systemie, dlatego częściej stosuje się mechanizmy oparte na uaktualnianiu cache po określonym czasie.}, aby zapewnić konsystencję między nim a bazą.
Innym problemem jest też konieczność dokonywania migracji schematu bazy danych wraz z rozwojem aplikacji.
Dodanie kolumny czy indeksu do tabeli zawierającej miliony wierszy czasem trwa nawet godzinami.

Wszystko powyższe wymaga bardzo dużych nakładów pracy na stworzenie i utrzymanie takiego systemu.
Co więcej, systemy stworzone według tej receptury tracą główne zalety relacyjnych baz danych: rezygnując z łączenia tabel rezygnujemy z relacyjności bazy, zaś wprowadzając shardowanie rezygnujemy z możliwości zadawania zapytań, które dotyczyłyby wszystkich wierszy tabeli.
W związku z tym, że systemy te są tworzone od zera na własne potrzeby i często są ściśle związane z infrastrukturą i architekturą aplikacji, bardzo problematyczne jest udostępnienie tego kodu na licencji open source.
W ostatnich latach jednak część z rozwiązań mających na celu zastąpić relacyjne bazy danych w pewnych rozwiązaniach wymagających skalowalności została upubliczniona.
Noszą one wspólną nazwę systemów NoSQL.

\section{NoSQL}
Powszechny i darmowy dostęp do produktów bazodanowych obiecujących wysoką skalowalność zaowocował powstaniem ruchu określanego popularnie mianem NoSQL (\emph{Not only SQL} - nie tylko SQL).
Termin ten powstał początkowo jako nazwa grupy użytkowników rozproszonych baz danych udostępnionych na licencji open source.
Ze względu na to, że nazwa ta umożliwia bardzo szeroką gamę interpretacji, pod metką NoSQL znajdujemy dziś systemy zaprojektowane w celu rozwiązywania diametralnie różnych problemów \cite{evans-nosql-what-is-in-a-name}. 
Ze względu na to, że większość nowych produktów bazodanowych stworzonych z myślą o wysokiej skalowalności nie dysponuje możliwością zadawania zapytań w języku SQL, niezależnie od tego czy są to bazy grafowe, oparte na dokumentach czy nawet proste kontenery klucz-wartość, wszystkie one noszą wspólną nazwę systemów NoSQL.

W literaturze trudno znaleźć nazwę problemu, który próbuje rozwiązać ruch NoSQL. 
W artykule \cite{monash-db-hvsp} autor sugeruje nazwę HVSP (ang. \emph{High Volume Simple Processing}). 
Sugerowane wyróżniki problemu to:
\begin{itemize}
 \item Wielu równocześnie korzystających z bazy użytkowników, dokonujących zarówno odczytów jak i zapisów.
 \item Operacje wykonywane przez użytkowników są nieskomplikowane, bez transakcji, łączenia tabel czy operacji grupujących.
\end{itemize}

Nie wszystkie systemy NoSQL spełniają powyższe warunki.
Warto pamiętać, że przeważająca większość tworzonych w dzisiejszych czasach systemów z powodzeniem działa rozproszona na co najwyżej kilka węzłów, często nawet bez potrzeby partycjonowania danych.
Dlatego właśnie, chociaż ruch NoSQL kojarzy się przede wszystkim ze skalowalnością, jednym z jego założeń jest dostosowanie systemu bazodanowego do potrzeb konkretnej aplikacji.
Przykładowo: chociaż w relacyjnej bazie danych można bez większego problemu odwzorować graf, bazy grafowe umożliwiają znacznie efektywniejsze implementacje algorytmów na tej strukturze danych, ponieważ złożoność operacji znalezienia krawędzi wychodzących bądź wchodzących do wierzchołka nie zależy od wielkości całego grafu (w odróżnieniu od bazy relacyjnej).

W dalszej części tej pracy opisane zostały różne rodzaje baz danych -- zarówno takie, które ułatwiają skalowanie aplikacji, jak i takie, które upraszczają różne aspekty modelowania danych. 
\chapter{Studium Literatury}

\section*{Streszczenie}

W studium literatury zajmę się opisaniem teorii CAP \cite{brewers-conjecture} (\emph{Consistency, Availability, Partition Tolerance}), zwanej również teorią Brewer'a od nazwiska jej autora. 
Teoria ta twierdzi, że rozproszony system nie jest w stanie zapewnić równocześnie wszystkich trzech gwarancji: konsystencji danych widzianych przez węzły systemu, odporności całości systemu na awarie poszczególnych jego węzłów oraz odporności systemu na utratę połączenia pomiędzy poszczególnymi węzłami lub ich grupami. 
Przedstawię także zasady działania Google BigTable \cite{google-bigtable} oraz Amazon Dynamo \cite{amazon-dynamo}, które bardzo istotnie wpłynęły na architekturę systemów NoSQL. 
Porównam także semantykę BASE (\emph{Basically Available, Soft-state, Eventually-consistent}) z ACID (\emph{Atomicity, Consistency, Isolation, Durability}).\todo{napisać dlaczego, uporządkować}

\section{Teoria CAP}

Teoria CAP (\emph{Consistency, Availability, Partition Tolerance}) zwana również teorią Brewera od nazwiska jej autora została po raz pierwszy zaprezentowana podczas prezentacji profesora Uniwersytetu Berkley Eryka Brewera 19 czerwca 2000r. na konferencji \emph{ACM Symposium on the Principles of Distributed Computing} \cite{podc-keynote}. 
W około dwa lata później, w 2002 roku, teoria ta (pod nazwą \emph{Brewer's Conjecture} - Domysł Brewera) została formalnie udowodniona przez Nancy Lynch oraz Setha Gilberta z MIT \cite{brewers-conjecture}.

Teoria CAP powstała jako efekt doświadczeń Brewera w firmie Inktomi oraz jego prac badawczych nad systemami rozproszonymi na Uniwersytecie w Berkley. 
Mówi ona, że z trzech pożądanych właściwości systemu rozproszonego: Konsystencji (ang. \emph{Consistency}), Wysokiej Dostępności (ang. \emph{Availability}) oraz Odporności na Podział Sieci (ang. \emph{Partition Tolerance}) możliwe jest zapewnienie co najwyżej dwóch z nich \cite{browne-cap-theorem}.

\subsection*{Konsystencja}

Termin ,,Konsystencja'' w Teorii CAP ma nieco inne znaczenie niż w ACID, gdzie oznacza on, iż zapisywane dane nie mogą złamać pewnych określonych reguł integralności. 
W Teorii CAP \emph{Consistency} jest dużo bardziej zbliżone do \emph{Atomicity} z ACID, oznacza ono bowiem, że gdy dokonamy operacji zapisu $x=x_0$ każdy kolejny odczyt $x$, niezależnie do którego węzła byłby skierowany, zwróci wartość $x_0$.

\subsection*{Wysoka Dostępność}

Wysoka dostępność jest najbardziej pożądaną właściwością z trzech wymienionych.
Oznacza ona, jak sama nazwa wskazuje, że system powinien udostępniać swoje usługi w pełni przez cały czas, wliczając w to awarie poszczególnych węzłów, aktualizacje oprogramowania czy awarie sieci. 
Bardziej formalnie: jeżeli operacja dotrze do nie ulegającego właśnie awarii węzła, to w pewnym skończonym czasie zwróci wynik do klienta.
Warto przy tym zwrócić uwagę (za  \cite{brewers-conjecture}), że dostępność zawodzi najczęściej właśnie wtedy, gdy jest najbardziej potrzebna, czyli w okresach największego obciążenia systemu.

\subsection*{Odporność na Podział Sieci}

W systemach rozproszonych, działających na tysiącach węzłów, często rozsianych w centrach obliczeniowych na wielu kontynentach utrata połączenia pomiędzy grupami węzłów jest oczekiwanym, codziennym problemem.
Podział sieci, rozumiany tu właśnie jako utrata połączenia między dowolnymi dwoma lub więcej węzłami systemu, przy zachowaniu połączenia tych węzłów z klientem, może nastąpić z wielu powodów: awarie switchy lub routerów, utrata połączenia między centrami obliczeniowymi, dokonywane naprawy.
W przypadku kiedy węzeł lub ich grupa zostanie całkowicie odcięty od klientów i innych węzłów systemu, możemy traktować te węzły tak samo jak gdyby na przykład odcięto im nagle zasilanie, dlatego definicja podziału takiego przypadku nie obejmuje.

Dla przykładu wyobraźmy sobie bazę danych replikowaną w systemie master-master.
Jeżeli połączenie między węzłami zostanie zerwane, modyfikacje dokonane na jednym z nich nie będą widoczne na drugim. 
Z kolei w sytuacji gdy mamy bazę danych z horyzontalnym podziałem danych (ang. \emph{sharded}), ponieważ każdy z serwerów zawiera informacje dotyczące tylko części danych i z góry wiadomo do którego z nich należy się zwrócić aby otrzymać informacje dotyczące dowolnego klucza, nawet w przypadku utraty połączenia między nimi, o ile klient ma dostęp do obu serwerów, ciągłość dostarczanych usług jest zapewniona.

\subsection*{Znaczenie Teorii}

Teoria CAP nabiera znaczenia w miarę wzrostu wielkości systemu. 
Gdy dysponujemy bazą rozsianą na kilku maszynach, narzut czasowy replikacji danych pomiędzy nimi jest akceptowalny dla większości zastosowań, nie musimy się też zbytnio martwić o podział sieci, gdyż zazwyczaj jest tak, że maszyny zlokalizowane w jednej szafie (ang. \emph{rack}) będą albo działać wszystkie, albo żadna. 
Kiedy jednak zajmujemy się usługami rozproszonymi na tysiącach węzłów, nawet gdybyśmy dysponowali 10000 maszynami o niezawodności MTBF 30 lat, każdego dnia następowałaby awaria którejś z nich \cite{google-lessons}. 
W przypadku tak dużych systemów, czas jakiego wymaga replikacja danych aby doprowadzić aby każdy węzeł widział ten sam stan, czy to jak system reaguje na podziały sieci nabiera o wiele większego znaczenia.

\subsubsection*{Podział systemów}

Teoria CAP mówi, że nie możemy zapewnić równocześnie wszystkich trzech gwarancji, dlatego skalowalne systemy muszą porzucić jedną z nich. 
Ze względu na to dzielimy je na:

\begin{enumerate}
 \item \emph{CA} - (Consistent, Available) te systemy mają problemy z podziałem sieci, wymagając zazwyczaj aby operacje dotyczące poszczególnych transakcji trafiały do pojedynczej grupy węzłów, które podlegają awarii ,,atomowo'' - albo wszystkie działają, albo żaden. 
 To podejście zazwyczaj wiąże się z problemami dla skalowalności.
 \item \emph{AP} - (Available, Partition-Tolerant) te systemy zapewniają największą odporność na awarie wynikające z rozproszonego środowiska, jednocześnie jednak stawiając twórców aplikacji przed trudnym zadaniem radzenia sobie z problemami wynikającymi z niespójności danych widzianych przez klientów bazy.
 \item \emph{CP} - (Consistent, Partition-Tolerant) te systemy w wypadku podziału oczekują na przywrócenie połączenia, ograniczając w ten sposób dostępność.
\end{enumerate}

W praktyce systemy omawiane w tej pracy zazwyczaj zadowalają się częściowym zapewnieniem wszystkich wymienionych gwarancji, przy czym wysoka dostępność odgrywa główną rolę, a poświęcana jest albo odporność na podziały, albo konsystencja. 
Dlatego systemy te należą albo do grupy CA (Google BigTable, HBase), albo do grupy AP (Amazon Dynamo, Riak).
\todo{Podział na CA, AP, CP - po lepszym poznaniu poszczególnych rozwiązań} 

\section{BASE}

BASE (\emph{Basically Available, Soft state, Eventual consistency}) to model konsystencji, który jest przeciwstawiany modelowi ACID (\emph{Atomicity, Consistency, Isolation, Durability}). 
Oba te modele dotyczą baz danych, zatem oba wymagają aby operacje klienta były trwałe (\emph{Durability}).
Różnica między nimi bierze się z podejścia do konsystencji.
W ACID system zawsze musi być w spójnym stanie, a dokonanie zmiany może wymagać szeregu operacji które doprowadzą system do tego stanu.
Operacje te ponadto zostają objęte transakcją i zostają aplikowane albo wszystkie, albo żadna.
W BASE system może być w stanie niespójnym z punktu widzenia aplikacji, a operacje które mają przywrócić tą spójność są wykonywane asynchronicznie.

Producenci relacyjnych baz danych od dawna już byli świadomi potrzeby partycjonowania danych na wiele węzłów.
Aby zapewnić semantykę ACID w kontekście rozproszonych transakcji stosuje się technikę 2PC (ang. \emph{2 Phase Commit}).
Protokół 2PC działa dwustopniowo:

\begin{enumerate}
 \item Najpierw koordynator transakcji żąda od wszystkich węzłów biorących udział w operacji aby wstępnie dokonały operacji commit dla transakcji i potwierdziły możliwość wykonania tej operacji.
 Jeżeli wszystkie węzły dokonały tego potwierdzenia, przechodzi się do drugiego kroku.
 \item W drugim kroku koordynator żąda od wszystkich zainteresowanych węzłów dokonania operacji commit.
 Jeżeli którakolwiek z baz zawetuje tą operację, wszystkie muszą wycofać transakcję.
\end{enumerate}

Problem, na jaki napotykamy w tym podejściu, to ograniczenie dostępności systemu (A w CAP).
Wystarczy aby jeden z węzłów systemu podległ awarii, aby cały system stał się niedostępny dla zapisów.
Dostępność w kontekście transakcji staje się iloczynem dostępności poszczególnych węzłów systemu.
Jeżeli mamy zatem węzły o indywidualnej dostępności 99.9\% to transakcja, która obejmuje trzy z nich będzie miała dostępność ok. 99.7\% - czyli o ok. 90 minut mniejszy \emph{uptime} w skali miesiąca \cite{base-an-acid-alternative}.

Jeżeli zatem ACID oferuje nam poziom konsystencji, który można byłoby określić mianem Strong Consistency, ale kosztem dostępności, to BASE oferuje w zamian wysoką dostępność kosztem konsystencji. 

\subsection{Eventual Consistency}

Eventual Consistency (w wolnym tłumaczeniu: konsystencja po pewnym czasie, ostatecznie) to słaba forma gwarancji konsystencji, która gwarantuje jedynie, że po pewnym, możliwym do przewidzenia czasie od momentu wykonania operacji, jej efekty będą widziane przez klientów systemu, niezależnie od tego, do którego z węzłów systemu się zwrócą z zapytaniem.
Okno czasowe między operacją a propagacją jej efektów do wszystkich zainteresowanych węzłów w systemie nazywamy oknem niespójności (ang. \emph{inconsistency window}).

Choć początkowo może się wydawać, że tego typu model sprawia duże trudności w implementacji aplikacji korzystających z baz danych, które go zapewniają, w rzeczywistości tego typu interakcje napotykamy każdego dnia.
Kiedy w systemie bankowym dokonujemy przelewu z konta na konto, pieniądze znikają z bilansu jednego z nich, ale na drugim pojawiają się z pewnym opóźnieniem.
Innym przykładem może być system DNS, gdzie zmiana jest propagowana w systemie stopniowo, często oczekując na przeterminowanie cache, ale po jakimś czasie jest zauważalna u każdego klienta.

Artykuł \cite{vogels-eventually-consistent} wprowadza ponadto kilka rodzajów Eventual Consistency:

\begin{enumerate}
 \item \emph{Causal Consistency} - Jeżeli proces A wykonał jakąś operację, a następnie zakomunikował ten fakt procesowi B, to proces B będzie widział zmienione dane w taki sam sposób jak proces A, a jego zapisy nie będą wchodzić w konflikt z tą operacją.
 Proces C, któremu ta informacja nie została przekazana, będzie podlegał normalnym regułom.
 \item \emph{Read-your-writes Consistency} - Proces dokonujący operacji w kolejnych operacjach zawsze widzi rezultaty tej operacji.
 \item \emph{Session Consistency} - To praktyczna realizacja poprzedniego rodzaju konsystencji.
 W tym przypadku proces komunikuje się z systemem w kontekście sesji, w ramach której ma zapewnioną gwarancję odczytu swoich operacji.
 W przypadku awarii i konieczności nawiązania nowej sesji, gwarancje te nie przechodzą na nowo nawiązaną sesję.
 \item \emph{Monotonic Write Consistency} - W tym przypadku system gwarantuje, że operacje zostaną wykonane w tej samej kolejności, w jakiej żądania ich wykonania zostały wysłane.
 Systemy które nie oferują tej gwarancji są bardzo trudne w użyciu.
\end{enumerate}

\subsubsection*{Konfiguracja Eventual Consistency}

Werner Vogels w artykule o Eventual Consistency \cite{vogels-eventually-consistent} wprowadził nomenklaturę stosowaną w konfiguracji konsystencji wielu systemów NoSQL (np. Cassandra, Riak).
Konfigurację tą opisują zazwyczaj trzy liczby:

\begin{enumerate}
 \item \emph{N} - liczba węzłów systemu na które zostanie zreplikowany pojedynczy rekord.
 Liczba ta jest zazwyczaj określana jako parametr konfiguracji systemu, lub podczas wydawania polecenia utworzenia ``tabeli'' (nazwanego zbioru rekordów).
 Niektóre systemy pozwalają na zmianę tej wartości w trakcie działania systemu (np. Riak), ale większość wymaga ponownego uruchomienia aplikacji w celu zastosowania tej zmiany.
 \item \emph{R} - liczba węzłów systemu, które muszą dokonać odczytu zanim wartość zostanie zwrócona do klienta.
 Czasem wartości zwrócone przez poszczególne węzły będą różne.
 Wtedy system  odpowiada albo za rozwiązanie konfliktu, albo za przekazanie wielu wersji klientowi.
 Parametr R jest najczęściej określany z osobna dla każdego polecenia odczytu.
 \item \emph{W} - liczba węzłów systemu, które muszą potwierdzić zapis aby operacja została zakończyła się sukcesem.
 Podobnie jak R jest to parametr przekazywany dla każdego zapytania.
\end{enumerate}

Jeżeli $R+W>N$, to mamy do czynienia z silną konsystencją (ang. \emph{Strong Consistency}) - każdy odczyt zwróci ostatnią zapisaną wartość.
Jeżeli $W = 0$, to zapis jest dokonywany w pełni asynchronicznie.

\subsubsection*{Znaczenie}

Eventual Consistency opisuje problem dotykający każdej rozproszonej bazy danych, niezależnie od tego czy jest to baza relacyjna, sieciowy system plików czy baza NoSQL.
Tradycyjne podejście do konsystencji w kontekście replikacji jest ograniczające.
Systemy bazodanowe najczęściej implementują rozwiązanie typu wszystko-albo-nic: operacja zapisu musi się powieść na wszystkich węzłach albo zostać cofnięta.
Podejście takie nie tylko ogranicza dostępność systemu, powoduje ono także ograniczenie możliwości decyzyjnych autorów aplikacji korzystających z tych systemów.
Nawet jeżeli system umożliwia asynchroniczną replikację, zazwyczaj jest to bardzo prymitywny mechanizm, który nie bierze pod uwagę wersjonowania rekordów i sprowadza bazę do konfiguracji R=1, W=1.

Z drugiej strony systemy takie jak Amazon Dynamo pozwalają swoim użytkownikom na pełną dowolność w konfiguracji mechanizmów persystencji.
Nawet przy $R+W>N$ mamy możliwość sterowania zachowaniem systemu: czy zapisy powinny być szybsze kosztem odczytów, czy na odwrót, czy też może gdzieś pomiędzy.
Wiele z tych systemów zapewnia \emph{Read-your-writes Consistency} co stanowi dodatkowe ułatwienie.
Możliwość konfiguracji parametrów R, W i N stopniowo staje się standardem wśród systemów NoSQL obsługujących partycjonowanie danych.

\section{Google MapReduce}

Google MapReduce \cite{google-mapreduce} jest biblioteką wykorzystywaną do przetwarzania dużych zbiorów danych w środowisku rozproszonym.
Użytkownik biblioteki specyfikuje dwie funkcje nazywane \emph{map} i \emph{reduce} oraz kilka innych parametrów konfiguracyjnych.
Następnie biblioteka dba o to aby dane wejściowe zostały podzielone, na poszczególnych rekordach została wykonana funkcja \emph{map}, a jej wyniki zostały zagregowane przy pomocy funkcji \emph{reduce}.

Operacje \emph{map} i \emph{reduce} są powszechnie spotykane w językach funkcyjnych, takich jak na przykład LISP.
\emph{Map} na wejściu otrzymuje parę $(k, v): k \in K_1, v \in V_1$ a na wyjściu emituje listę par $(k, v): k \in K_2, v \in V_2$.
\emph{Reduce} na wejściu otrzymuje parę $(k, (v_1, ..., v_n)): k \in K_2, v_1...v_n \in V_2$ i na wyjściu emituje $v: v \in V_2$.

\subsection*{Przykład}

Poniżej przedstawiam przykładowy kod funkcji \emph{map} i \emph{reduce} w języku Python dla problemu zliczania wystąpień słów w zbiorze tekstów. 
Jak widzimy funkcja \emph{map} przyjmuje pary (klucz, wartość) z przestrzeni (Nazwy Dokumentów, Treść Dokumentów), zwracając pary z innej przestrzeni (Słowa, Liczby Naturalne).
Funkcja \emph{reduce} w przykładzie przyjmuje pary gdzie kluczem jest słowo, natomiast wartością jest lista liczb naturalnych określająca liczby wystąpień tego słowa w różnych dokumentach (albo wielokrotnie w tym samym dokumencie).

\begin{verbatim}
def map(document_name, document_value):
  """ funkcja mapujaca dokumenty na pary (slowo, 1)  """
  for word in words(document_value):
    yield (word, 1) # emit word

def reduce(word, counts):
  """ 
  funkcja redukujaca, przyjmuje slowo 
  i liste (iterator) po liczbie jego wystapien
  zwrace sumaryczna liczbe wystapien danego slowa
  """
  sum = 0
  for value in counts:
    sum += value
    # sum += 1
    # tak tez mozna byloby zapisac, 
    # ale wtedy funkcja nie bylaby laczna     
  return sum 
\end{verbatim}


\subsection*{Opis działania}

Użytkownik tworzy aplikację, w której specyfikuje dwie funkcje \emph{map} i \emph{reduce}, oraz dwa parametry konfiguracyjne: \emph{M} i \emph{R}:

\begin{itemize}
 \item \emph{M} - określa na ile części ma zostać podzielony plik wejściowy.
 Zazwyczaj wybiera się taką liczbę aby wielkość plików wejściowych zawierała się między 16MB a 64MB.
 Ponieważ GFS dzieli pliki na kawałki (ang. \emph{chunks}) wielkości 64MB, jest dość istotne aby pliki wejściowe nie przekraczały tej wielkości, gdyż w przeciwnym przypadku mogłaby występować konieczność komunikacji sieciowej w celu odczytania danych z pliku wejściowego.
 \emph{M} określa ponadto liczbę zadań \emph{map}.
 \item \emph{R} - określa liczbę zadań \emph{reduce}.
\end{itemize}

\myfigure{chapters/studium_literatury/map-reduce.png}{Google MapReduce}{fig:map-reduce}

\begin{enumerate}
 \item Biblioteka MapReduce dzieli plik wejściowy na \emph{M} części.
 \item Program użytkownika zostaje wysłany i uruchomiony na maszynach klastra.
 Jedna z tych maszyn przyjmuje specjalną rolę \emph{master}, pozostałe zaś mają rolę \emph{worker}\todo{może powinienem zamienić master na zarządca/nadzorca a worker na pracownik?}.
 \item \emph{master} przypisuje poszczególnym \emph{workerom} po jednym zadaniu do wykonania.
 Kolejne są przydzielane w miarę jak węzły kończą przydzieloną im pracę.
 Zadanie \emph{map} zostanie przydzielone w pierwszej kolejności maszynie która jest równocześnie \emph{chunkserverem} przechowującym odpowiedni plik wejściowy.
 Pozwala to uniknąć komunikacji sieciowej w celu odczytania pliku.
 \item \emph{Worker} przetwarza plik wejściowy rekord po rekordzie wywołując funkcję \emph{map} i zapisując jej wynik w pamięci.
 \item Co pewien czas dane zapisane w pamięci są zrzucane na dysk do plików lokalnych.
 W tym procesie dane są rozdzielane do \emph{R} plików poprzez funkcję partycjonującą, domyślnie $hash(key) mod R$.
 Lokacje tych plików są przekazywane do węzła \emph{master}, który z kolei jest odpowiedzialny za przekazanie ich do węzła wykonującego operację \emph{reduce} na odpowiednim fragmencie danych.
 \item Węzeł wykonujący operację \emph{reduce} po otrzymaniu takiego powiadomienia pobiera odpowiednie pliki bezpośrednio od węzła, który je przechowuje.
 Po otrzymaniu wszystkich potrzebnych plików, \emph{worker} sortuje otrzymane dane po kluczu, tak aby wartości dla danego klucza sąsiadowały ze sobą w pliku.
 \item Po posortowaniu plików wejściowych węzeł iteruje po kluczach i dla każdego z nich przekazuje klucz oraz listę wszystkich przypisanych mu wartości do funkcji \emph{reduce}, zapisując następnie jej wynik w pliku wyjściowym.
 \item Kiedy wszystkie operacje \emph{reduce} zakończą się, \emph{master} budzi program użytkownika i wywołanie funkcji \emph{MapReduce} kończy się.
\end{enumerate}

Wynikiem operacji jest \emph{R} plików wynikowych.
W większości przypadków konsumentem tych danych są inne operacje MapReduce, bądź aplikacje rozproszone, więc nie ma potrzeby łączenia tych plików w jedną całość.

\subsection*{Optymalizacje}

W artykule \cite{google-mapreduce} zostało opisanych kilka istotnych optymalizacji i ulepszeń:

\begin{enumerate}
 \item Operacja \emph{map} na pliku wejściowym jest wykonywana na tym samym serwerze, który przechowuje ten plik.
 \item Operacja \emph{map} może zwrócić bardzo wiele wartości dla danego klucza pośredniego.
 Z tego względu biblioteka wprowadza pojęcie funkcji łączącej (ang. \emph{combiner}).
 Funkcja ta dokonuje wstępnej redukcji przed wysłaniem wartości przez sieć do węzła wykonującego operację \emph{reduce}.
 Najczęściej stosuje się w tym miejscu tą samą funkcję co w operacji \emph{reduce}, ale aby to było możliwe, funkcja ta musi być łączna i przemienna, co czasem wymaga wprowadzenia pewnych zmian.
 Dla przykładu: przedstawiona wcześniej funkcja map emitująca wszystkie słowa w danym dokumencie emituje powtarzające się słowa wielokrotnie.
 Poprzez zsumowanie wystąpień przed przesłaniem znacznie zmniejszamy ilość danych do wysłania.
 \item Funkcja partycjonująca może być wyspecyfikowana przez użytkownika, dzięki czemu potencjalnie możliwe jest takie jej określenie, aby dane powiązane ze sobą znalazły się w jednym pliku wynikowym (aczkolwiek kosztem ryzyka nierównomiernego podziału kluczy między partycje).
 \item Dzięki sortowaniu kluczy pośrednich, pliki wynikowe są łatwiejsze do przetwarzania, umożliwiając na przykład wyszukiwanie binarne.
 \item Kiedy zbliża się koniec operacji, zadania \emph{reduce} są zlecane do wykonania przez dodatkowe węzły.
 Dzięki temu uszkodzone, nadmiernie obciążone przez inne procesy albo wadliwie skonfigurowane maszyny nie powodują nadmiernego wydłużenia całości operacji MapReduce.
\end{enumerate}

\subsection*{Odporność na awarie}

Ponieważ biblioteka MapReduce służy do wykonywania operacji w rozproszonym środowisku, często nawet na tysiącach węzłów, konieczne jest aby była ona odporna na awarie części z węzłów.
Rozróżniamy dwa typy awarii: awaria węzła \emph{master} i awarie węzłów typu \emph{worker}.

\subsubsection*{Awaria węzła master}

Ponieważ \emph{master} przechowuje między innymi informacje o zrealizowanych zadaniach i lokalizacji plików wynikowych.
\emph{Master} musi działać aby operacja MapReduce się zakończyła powodzeniem, ale możliwe jest aby jego struktury danych były zapisywane w GFS, albo replikowane do zapasowego węzła.
W opisanym systemie awaria węzła master zawsze kończy się niepowodzeniem całości operacji MapReduce i koniecznością jej ponownego uruchomienia.
Jest to dopuszczalne ponieważ czas trwania operacji jest liczony w minutach, więc awaria tego konkretnego węzła jest mało prawdopodobna (w odróżnieniu od np. BigTable, który jest systemem, który działa bez przerwy).

\subsubsection*{Awaria węzła slave}

\emph{Master} regularnie komunikuje się z węzłami \emph{slave} w celu ustalenia ich stanu.
W przypadku gdy węzeł przestaje odpowiadać jest uznawany za ,,martwy'', w związku z czym \emph{master} oznacza zadanie aktualnie przez niego wykonywane jako przeznaczone to przydziału, tak samo wszystkie wykonane przez niego zadania \emph{map}.
Zadania \emph{map} muszą być wykonane ponownie, ponieważ ich wyniki zostały zapisane w lokalnych plikach i przez to są niedostępne do odczytu.
Zadania \emph{reduce} zapisują swoje wyniki w GFS, w związku z tym nie muszą być powtarzane. 

\subsection*{Znaczenie}

Większość systemów opisanych w niniejszej pracy nie dysponuje zaawansowanymi metodami wykonywania zapytań, a w szczególności zapytań agregujących czy zliczających takich jak funkcje COUNT, SUM, AVG i operator GROUP BY w relacyjnych bazach danych.
Operacje tego typu są szczególnie trudne w kontekście systemów rozproszonych, gdzie wykonanie zapytań tego typu w trybie on-line przy zachowaniu odpowiedniego czasu odpowiedzi jest często wręcz niemożliwe.
Dlatego najczęściej wykonywanie tego typu operacji jest dokonywane co pewien czas, lub zlecane przy zapisie danych, a jego wyniki są przechowywane w systemie.

W przeważającej większości obliczenia te są wykonywane przy zastosowaniu algorytmu MapReduce.
Wiele systemów, tak jak MongoDB czy Riak, zapewnia taką funkcjonalność, inne wymagają zastosowania zewnętrznych narzędzi.
Wierną implementacją Google MapReduce jest Hadoop, który opiszemy nieco bliżej przy okazji Hbase.
Bardzo ciekawe podejście do MapReduce prezentuje CouchDB, która przechowuje wyniki funkcji map, dzięki czemu umożliwia dokonywanie zapytań opartych o MapReduce w trybie on-line.
CouchDB zostanie opisane bliżej w kolejnym rozdziale.

Algorytm MapReduce ma duże znaczenie w systemach NoSQL, gdyż stanowi surogat dla wykonywania skomplikowanych zapytań.
Stanowi on jedno z podstawowych narzędzi dla praktyki prostych odczytów, ale złożonych zapisów.

\section{Amazon Dynamo}

Amazon Dynamo to baza typu klucz-wartość używana w największym na świecie sklepie internetowym \emph{amazon.com}.
Baza została opisana w artykule z 2007 roku \cite{amazon-dynamo} i od tego czasu opisany w nim system doczekał się już dwóch implementacji open-source: Dynomite i Riak, oraz stał się inspiracją dla innych baz takich jak Cassandra, czy rozszerzenia CouchDB pozwalającego na rozpraszanie tej bazy.

Amazon Dynamo skaluje się bardzo dobrze - do setek węzłów, co dzięki zastosowaniu architektury opartej na usługach w zupełności wystarcza nawet takiemu gigantowi jak Amazon.
Interesującym aspektem architektury tego sklepu internetowego jest to, że poszczególne usługi muszą oferować gwarancje na czas wykonania (ang. \emph{Service Level Agreements}).
Przykładowym kontraktem tego typu jest gwarancja, że usługa zwróci odpowiedź w ciągu 300ms dla 99,9\% zapytań przy obciążeniu 500 zapytań na sekundę.
Konsekwencją istnienia takich gwarancji jest to, że Dynamo musi udostępniać wiele możliwości konfiguracji, tak aby usługi mogły dostosować właściwości bazy na potrzeby oferowanego kontraktu.

Jednym z wymagań dla systemu obsługującego sklep internetowy jest to aby użytkownik był w stanie zawsze dodać przedmioty do koszyka czy złożyć zamówienie.
W związku z tym Amazon Dynamo jest systemem w którym dostępność (A w CAP) odgrywa kluczową rolę, a odporność na podział sieci (P w CAP) jest jej uzupełnieniem.
Wydawało by się, że konsystencja ma duże znaczenie w przypadku sklepu internetowego - nie można przecież sprzedać towaru, którego się nie ma, a przecież w przypadku sklepu z którego korzystają miliony użytkowników równocześnie warunki wyścigu (ang. \emph{race conditions}) muszą występować nagminnie.
W amazon.com dopuszczalne jest aby dwóch klientów zamówiło ostatnią z książek w magazynie, a kiedy okaże się że jeden z nich jej nie może otrzymać sklep kontaktuje się z nim proponując na przykład dłuższy okres dostawy.
Nawet jeżeli nie wszyscy użytkownicy się zgodzą z taką sytuacją, to i tak przekłada się to na wyższe przychody.
Pokazuje to jak Eventual Consistency może z powodzeniem być stosowane w sytuacjach, gdzie normalnie kładzie się duży nacisk na spójność danych, nie powodując przy tym ujmy dla systemu.

\subsection*{Główne założenia architektury}

Autorzy Amazon Dynamo określili takie założenia co do architektury:

\begin{itemize}
 \item system jest zawsze w stanie przyjąć operację zapisu
 \item w związku z replikacją danych i naciskiem na wysoką dostępność, w systemie może wystąpić wiele wersji rekordu
 \item rozwiązywanie konfliktów między wersjami następuje przy odczycie i jest dokonywane przez aplikację, nie przez bazę (ale aplikacja może z tego zrezygnować i wybrać prosty mechanizm, np. ostatni zapis wygrywa)
 \item system musi się łatwo skalować wszerz poprzez dodawanie kolejnych węzłów i nie powinno to mieć dużego wpływu ani na administrację systemem ani na sam system
 \item system musi się charakteryzować symetrią: każdy węzeł ma taki sam zestaw obowiązków jak wszystkie inne
 \item system musi być zdecentralizowany: żaden węzeł nie jest wyróżniony i wszelkie operacje są wykonywane korzystając z mechanizmów \emph{peer-to-peer}
 \item system musi radzić sobie z heterogenicznością środowiska: narzut pracy na pojedynczy węzeł powinien uwzględniać jego możliwości w porównaniu do innych węzłów
\end{itemize}

\subsection*{Stosowane techniki}

Amazon Dynamo łączy w sobie wiele ciekawych technik rozwiązujących problemy takie jak partycjonowanie, wersjonowanie rekordów czy wykrywanie awarii.
W niniejszym rozdziale postaram się je czytelnikowi przybliżyć.

\subsubsection*{Consistent Hashing} 

W systemie takim jak Dynamo, który ma za zadanie skalować się do setek węzłów, horyzontalne partycjonowanie jest nieuniknione.
W Amazon Dynamo za podział na partycje odpowiada zmodyfikowany algorytm Consistent Hashing.
W następnych kilku paragrafach opiszę najpierw podstawowy algorytm, a następnie kolejne jego modyfikacje.

Consistent Hashing to algorytm który dla dowolnego klucza określa który węzeł systemu go przechowuje.
Ponieważ klucze są dowolnymi ciągami bajtów, zawsze operując na nich używamy funkcji mieszającej (ang. \emph{hashing function}).
Przeciwdziedzinę tej funkcji mieszającej traktujemy jako pierścień gdzie najmniejsza i największa możliwa wartość niejako stykają się ze sobą, podobnie jak ma to na przykład miejsce w przypadku długości geograficznych czy godzin na tarczy zegara.
W podstawowej wersji systemu każdy węzeł systemu losuje jedną wartość na tym pierścieniu.
Aby określić któremu węzłowi odpowiada dany klucz, obliczamy dla niego wartość funkcji mieszającej, a następnie znajdujemy węzeł, którego ,,pozycja'' na pierścieniu jest najbliższą zgodnie z ruchem wskazówek zegara po wartości wyliczonej.
W takim schemacie każdemu węzłowi odpowiada jeden, ciągły fragment pierścienia (patrz rysunek \ref{fig:consistent-hashing-01}).

\myfigure{chapters/studium_literatury/consistent-hashing-01.png}{Podstawowe Consistent Hashing}{fig:consistent-hashing-01}

Podstawowy wariant algorytmu ma wiele wad.
Problemem jest na przykład potencjalnie bardzo nierównomierna dystrybucja kluczy między węzłami.
Nie ma też żadnego wsparcia dla heterogeniczności środowiska - nie możemy ,,mocniejszemu'' serwerowi przekazać większego zakresu kluczy.
W związku z tymi ograniczeniami, wprowadza się tak zwane ,,wirtualne węzły'': pojedynczemu fizycznemu węzłowi przypisuje się wiele pozycji na pierścieniu (zazwyczaj liczba tych pozycji jest znacznie większa od ogólnej liczby węzłów).
Dzięki tej modyfikacji (o ile liczba wirtualnych węzłów jest wystarczająco duża) dystrybucja kluczy będzie znacznie bardziej sprawiedliwa, możemy też kontrolować obciążenie poszczególnych węzłów poprzez przyporządkowanie im mniejszej lub większej liczby pozycji na pierścieniu (patrz rysunek \ref{fig:consistent-hashing-02}).

\myfigure{chapters/studium_literatury/consistent-hashing-02.png}{Consistent Hashing z wirtualnymi węzłami}{fig:consistent-hashing-02}

Opisana powyżej wersja algorytmu radzi sobie zbyt dobrze z dodawaniem i usuwaniem węzłów niż wersja podstawowa, jednak w rzeczywistości nie sprawdza się zbyt dobrze.
Przy dodaniu węzła do systemu, klucze znajdujące się do tej pory pod opieką innych węzłów muszą być przekazane nowemu węzłowi.
Aby tego dokonać, konieczne jest dokonanie przeglądu wszystkich kluczy w dzielonym zakresie, konieczne jest także przeliczenie na nowo i synchronizacja drzew Merkle (vide \emph{Merkel Trees} na stronie \pageref{merkle-trees}). 
Przegląd taki obciąża system i musi być dokonywany w tle aby nie złamać gwarancji dawanych przez korzystającą z Dynamo usługę.
Aby uniknąć tego problemu należy rozdzielić mechanizm partycjonowania danych (w jaki sposób są tworzone przedziały) od tego które węzły nimi zarządzają.

W tym wariancie algorytmu przestrzeń kluczy zostaje podzielona z góry na określoną liczbę przedziałów S.
Najczęściej liczba ta będzie znacznie większa niż liczba węzłów systemu N ($S >> N$).
Następnie każdy z przedziałów przydzielamy losowo jednemu z węzłów tak aby każdy miał ich po $S/N$\footnote{oczywiście ten mechanizm może być dostosowany do wymagania heterogeniczności, tak aby jedne węzły otrzymywały więcej przedziałów niż inne}.
Dzięki temu rekordy przynależące do różnych z przedziałów mogą być przechowywane w odrębnych plikach, które z kolei mogą być przesyłane w ramach potrzeby do przejmujących nad przedziałem kontrolę węzłów bez zbędnego przeglądania danych na dysku.
Rysunek \ref{fig:consistent-hashing-03} przedstawia opisaną powyżej wersję algorytmu.

\myfigure{chapters/studium_literatury/consistent-hashing-03.png}{Consistent Hashing z rozdziałem partycjonowania i pozycjonowania rekordów}{fig:consistent-hashing-03}

\subsubsection*{Merkle Trees}
\label{merkle-trees}

\section{Google BigTable}

W kolejnym rozdziale opiszę Google BigTable \cite{google-bigtable} - bazę stosowaną w Google, przewyższającą Amazon Dynamo o rząd wielkości pod względem możliwej liczby węzłów, opartą na rozproszonym systemie plików GFS.
Google BigTable jest przykładem bazy CA (\emph{Consistent, Avaliable}).
W rozdziale opisującym dostępne systemy opiszę Hbase - bazę silnie wzorowaną na BigTable, ale opartą na Hadoop Core zamiast GFS, oraz Apache Cassandra, która łączy w sobie cechy Amazon Dynamo i Google BigTable.

\chapter{Klasyfikacja rozwiązań}

\section*{Streszczenie}

Jednym z założeń ruchu NoSQL jest zwrócenie uwagi użytkowników systemów bazodanowych na fakt, że dla różnych aplikacji i do różnych zastosowań mogą pasować zupełnie różne systemy.
Ogromna liczba nierelacyjnych baz danych\footnote{Strona http://nosql-database.org/ wymienia ponad stu takich baz.} sprawia, że niemożliwym jest opisanie wszystkich z nich w niniejszej pracy.
Z tego względu przedstawionych zostanie jedynie kilka najlepiej znanych i najbardziej popularnych rozwiązań.

Istnieje wiele możliwych sposobów klasyfikacji systemów NoSQL.
Na potrzeby tej pracy przyjęto najczęściej stosowany system, który dzieli bazy ze względu na wykorzystywany model danych.

\section{Podział ze względu na model danych}

\begin{description}
 \item[Bazy typu klucz-wartość]
 Systemy tego typu modelują dane jako tablicę asocjacyjną.
 W takiej tablicy unikalnemu kluczowi jest przyporządkowana wartość.
 Zazwyczaj ani klucz, ani jego wartość nie są interpretowane przez system.
 Bazy tego typu udostępniają zazwyczaj jedynie prosty interfejs pozwalający na odczytanie lub ustawienie wartości klucza bądź jego usunięcie; bardziej skomplikowane zapytania są zazwyczaj niemożliwe.
 
 \item[Bazy kolumnowe]
 Systemy te wzorują swój model danych na Google BigTable (patrz strona \pageref{google-bigtable-model-danych}).
 W bazie kolumnowej unikalnemu kluczowi odpowiada wiele kolumn.
 Kolumny są grupowane w rodziny - zazwyczaj wszystkie kolumny należące do tej samej rodziny są jednego typu.
 W przypadku Apache Cassandra kilka rodzin kolumn może być zebranych w super-rodzinę.
 Kolumny przynależące do jednej rodziny z reguły zawierają dane jednego typu, gdyż są one kompresowane razem.
 Bazy te zazwyczaj wymagają zdefiniowania struktury bazy z dokładnością do rodzin, czy super-rodzin kolumn.
 Jedynie kolumny, którym jest przypisana jakaś wartość dla danego klucza zajmują miejsce w pamięci, dzięki temu różnych kolumn może być bardzo wiele i często służą na przykład do tworzenia list wartości.
 Bazy kolumnowe pozwalają jedynie na wykonywanie zapytań o pojedynczy klucz lub ich zakres, ale za to oferują szeroki zakres możliwości limitowania zwracanych kolumn.

 \item[Bazy dokumentowe]
 W systemie dokumentowym unikalnemu kluczowi przypisany jest tak zwany dokument.
 Dokument jest to drzewo, którego liśćmi są wartości prymitywne (np. ciąg znaków, liczba, wartość logiczna, pusty obiekt NULL, itp.) a węzłami wartości złożone (tablica, tablica asocjacyjna).
 Dokumenty są zazwyczaj zgrupowane w kolekcje, po których można iterować.
 Dokumenty należące do tej samej kolekcji nie muszą mieć tych samych pól czy struktury.
 W odróżnieniu od systemu typu klucz-wartość przechowującego dane w formacie JSON, systemy dokumentowe są w stanie interpretować wartości zawarte w dokumentach i umożliwiają użytkownikowi efektywne wyszukiwanie rekordów na podstawie tych wartości przy pomocy różnych rodzajów indeksów.
 W bazie dokumentowej nie występują zazwyczaj relacje między dokumentami, ale możliwe jest zagnieżdżanie dokumentów, co ułatwia modelowanie relacji jeden-do-wielu.
 
 \item[Bazy grafowe]
 Głównym wyróżnikiem grafowych baz danych jest to, że baza grafowa pozwala w złożoności O(1) uzyskać informacje o relacjach danego węzła z innymi węzłami.
 W takim systemie można uzyskać informacje zarówno o krawędziach wchodzących jak i wychodzących.
 Zarówno węzły, jak i krawędzie grafu są opisane właściwościami.
 Podobnie jak w przypadku baz dokumentowych węzły czy krawędzie nie mają z góry narzuconej struktury i mogą się różnić liczbą, typem i nazwami właściwości.
 Podstawowym mechanizmem zapytań w bazie grafowej są rożne mechanizmy trawersowania grafu biorąc pod uwagę właściwości węzłów i krawędzi i poruszając się po krawędziach wchodzących i wychodzących.
 Drugorzędnym mechanizmem jest wyszukiwanie węzłów i krawędzi grafu na podstawie ich właściwości.
 Możliwość wykonywania takich zapytań nie jest jednak konieczna w bazie grafowej i niektóre produkty pozostawiają obsługę takich zapytań innym systemom. 

 \item[Skalowalne bazy relacyjne]
 Do tej grupy należą systemy, które zachowują relacyjny model danych, oraz najczęściej SQL jako język zapytań, ale w odróżnieniu od typowych relacyjnych baz danych są zaprojektowane z myślą o horyzontalnej skalowalności.
\end{description}

\section{Opisane aspekty rozwiązań}

Ponieważ dokładne przedstawienie dowolnego systemu bazodanowego na kilku stronach A4 jest zadaniem nie do wykonania, zamierzeniem autora tej pracy było wybranie najbardziej istotnych i cennych dla czytelnika informacji.
Pominięte zostały przede wszystkim szczegóły dotyczące API poszczególnych systemów oraz łatwo dostępne informacje, takie jak instrukcja instalacji systemu.
Opisane zostały natomiast istotne cechy systemów, które pozwalają czytelnikowi lepiej zrozumieć mechanizm działania aplikacji.
Każda baza została opisana według następującego schematu:

\begin{description}
 \item[Wstęp] 
 Każdą sekcję poprzedzono wstępem przybliżającym w kilku zdaniach opisywaną bazę. 
 
 \item[Protokół komunikacji]
 Opisuje dostępne protokoły komunikacji z bazą i wsparcie dla różnych języków programowania.

 \item[Replikacja]
 Opisuje dostępne mechanizmy replikacji systemu i możliwości ich konfiguracji.

 \item[Partycjonowanie]
 W przypadku baz obsługujących horyzontalne partycjonowanie danych opisany zostanie ten mechanizm i jego możliwości konfiguracji.
 W przypadku baz nie dających takiej możliwości opisane zostaną narzędzia ją zapewniające.

 \item[Persystencja]
 Opisany zostanie mechanizm trwałego zapisu danych oraz różne opcje konfiguracji systemu z tym związane i w przypadku systemów, które obsługują wymienne moduły persystencji, zostaną one opisane.

 \item[Wersjonowanie]
 W przypadku systemów które oferują taką opcję, zostanie opisany mechanizm wersjonowania rekordów i rozwiązywania konfliktów.

 \item[Wyszukiwanie]
 Opisane zostaną ogólnie opcje tworzenia zapytań do systemu i indeksowania danych.

 \item[Unikalne cechy]
 Przedstawione zostaną cechy systemu, które wyróżniają go na tle konkurencji.

 \item[Typowe zastosowania]
 Opisane zostaną typowe zastosowania i znane wdrożenia danego systemu.

 \item[Przeciwwskazania]
 Opisuje typy zastosowań, z którym system może mieć problemy, szczególnie takie, które mogą nie występować przy prostej ewaluacji.

 \item[Dokumentacja i wsparcie]
 Opisany zostanie poziom aktywności społeczności związanej z daną bazą, możliwości komercyjnego wsparcia, subiektywna ocena jakości dokumentacji oraz w przypadku produktów rozwijanych przez konkretnych dostawców, zostaną oni przedstawieni.

 \item[Pomocne odnośniki]
 Każdy opis zostanie zakończony garścią odnośników pomocnych w zapoznawaniu się z systemem.

\end{description} 
\chapter{Bazy Klucz-Wartość}

\section*{Streszczenie}
W tym rozdziale opisane zostaną dwie bazy klucz-wartość: Redis i Riak.

\section{Redis}
\label{sec:redis}

\section{Riak}
\label{sec:riak}
\chapter{Bazy Kolumnowe}

\section*{Streszczenie}
W tym rozdziale opisane zostaną dwie bazy kolumnowe: HBase i Apache Cassandra.
Mimo podobnego modelu danych, bazy te znacznie różnią się pod wieloma względami.
HBase powstało jako implementacja open source systemu opisanego w artykule o Google BigTable i mimo ciągłego rozwoju i dodawania coraz nowszych funkcjonalności te dwa systemy są do siebie wciąż bardzo zbliżone.
Z drugiej strony Apache Cassandra, który to system został zaprojektowany między innymi przez jednego z autorów Amazon Dynamo, mimo że czerpie swój model danych z Google BigTable, na poziomie architektury znacznie bardziej przypomina system firmy Amazon niż Google, a co za tym idzie podobnie jak Riak należy do grupy systemów oferujących bardzo duże gwarancje dostępności.

\section{Apache Cassandra}
\label{sec:cassandra}

\subsection*{Wstęp}

Apache Cassandra powstała na potrzeby jednej z największych stron internetowych na świecie: sieci społecznościowej Facebook. 
Zadaniem tego systemu było zastąpienie wcześniejszej architektury opartej o MySQL do wyszukiwania w skrzynkach wiadomości użytkowników.
Jednym z architektów nowego systemu był Avinash Lakshman, który wcześniej był współautorem Amazon Dynamo.
W 2008 roku Facebook zdecydował się na udostępnienie swojego produktu na licencji open source.
Przez pierwszy rok jedynie programiści firmy Facebook mieli możliwość zmiany kodu aplikacji, co w połączeniu z tym, że odmawiali oni przyjmowania zmian autorstwa innych programistów, powodowało duże napięcia w społeczności powiązanej z tym projektem i podniesienie się licznych głosów nawołujących do podziału projektu (ang. \emph{fork}).
W marcu 2009 Facebook przekazał prawa do Cassandra fundacji Apache, która okazała się dużo bardziej sprawna w zarządzaniu tym projektem\cite{evans-cassandra}.
W tym samym roku ukazała się także publikacja\footnote{Czytając ten artykuł warto jednak pamiętać, że odnosi się on do systemu w postaci w jakiej został on stworzony na potrzeby firmy Facebook, a nie do systemu, który ostatecznie został upubliczniony i nosi teraz nazwę Apache Cassandra. Główną różnicą między nimi jest to, że integracja z systemem Zookeeper nie została upubliczniona i nie jest częścią Apache Cassandra, a co za tym idzie różnią się one tym, że dostępna publicznie wersja systemu nie ma wyróżnionych węzłów i nie zależy od innych systemów.} omawiająca architekturę projektu autorstwa jego twórców \cite{cassandra-paper}.

Apache Cassandra stanowi bardzo ciekawe połączenie modelu danych zaczerpniętego z Google BigTable z modelem replikacji i partycjonowania (a co za tym idzie konsystencji) zaczerpniętym z Amazon Dynamo.
Podobnie jak opisany wcześniej Riak, Cassandra nie posiada wyróżnionych węzłów, ale w odróżnieniu od tej bazy typu klucz-wartość posiada ona bardziej zaawansowany model danych.

\subsection*{Protokół komunikacji}

Apache Cassandra posiada API wykorzystujące technologię RPC Apache Thrift\footnote{Apache Thrift został stworzony na potrzeby firmy Facebook. Jest on podobny do technologii CORBA.} do komunikacji, dzięki czemu oferuje wsparcie dla 12 języków programowania.
Dla najpopularniejszych języków programowania takich jak Java, Python czy Ruby istnieją także biblioteki wyższego poziomu, często wzorowane na bibliotekach ORM.

Cassandra pozwala na rozszerzanie podstawowego zestawu funkcjonalności systemu (na przykład zestawu komparatorów używanego do sortowania) poprzez implementację własnych klas w języku Java (w tym języku została napisana ta baza).
Dzięki temu, że istnieje wiele języków programowania, które są kompilowane do kodu pośredniego maszyny wirtualnej Java (JVM), rozszerzenia te można pisać także w wielu innych językach.

\subsection*{Replikacja}

Apache Cassandra czerpie swój model replikacji z Amazon Dynamo.
Podobnie jak w przypadku tamtej bazy użytkownik specyfikuje parametry R, W i N, określające odpowiednio ile węzłów musi odpowiedzieć na zapytanie aby odczyt się powiódł, ile musi potwierdzić zapis, oraz na ile węzłów jest replikowany dany klucz.

Bardzo ciekawą właściwością tej bazy jest to, że strategia wyboru węzłów, na które zostanie replikowany dany przedział kluczy może być konfigurowana.
Poza domyślną strategią, która podobnie jak w Amazon Dynamo replikuje klucze na N-1 kolejnych węzłów, dostępne są także dwie dodatkowe strategie: ,,Rack Aware Strategy'' oraz ,,Data Center Shard Strategy''.
Pierwsza z nich sprawia, że dla każdego przedziału drugi węzeł na liście preferencyjnej\footnote{patrz strona \pageref{sec:dynamo-replikacja}.} będzie węzłem z innego centrum obliczeniowego, a kolejne będą z innej szafy (ang. \emph{rack}) niż koordynator.
Druga z dodatkowych strategii pozwala na zdefiniowanie jak repliki mają zostać podzielone między centrami obliczeniowymi w przypadku gdy jest ich więcej niż dwa.
Dla przykładu w przypadku gdy dysponujemy trzema centrami obliczeniowymi, a każdy klucz jest replikowany na sześć węzłów, możemy zdefiniować że trzy z nich będą węzłami z pierwszego centrum, dwa z drugiego i jeden z trzeciego.

Podobnie jak Riak, Cassandra pozwala na specyfikowanie parametrów R i W dla zapytań nie tylko jako liczby, ale także jako wartości symboliczne, np. $ALL$ (wszystkie), czy $QUORUM$ (większość).
Dodatkowo dostępna jest wartość $DCQUORUM$, która działa w połączeniu ze strategiami opisanymi powyżej.
Pozwala ona na wyspecyfikowanie, że jedynie węzły z lokalnego centrum obliczeniowego mają być brane pod uwagę przy określaniu liczby węzłów potrzebnej do osiągnięcia kworum, dzięki czemu można zapewnić konsystencję na poziomie centrum danych oraz uniknąć oczekiwania na odpowiedź odległych węzłów.

\subsection*{Partycjonowanie}

Cassandra wykorzystuje algorytm \emph{Consistent Hashing}\footnote{patrz strona \pageref{sec:dynamo-consistent-hashing}.} w wersji podstawowej.
Każdemu węzłowi jest przyporządkowany jeden przedział kluczy, których jest koordynatorem, a ponadto każdy klucz jest replikowany na N-1 kolejnych węzłów.

Algorytm podziału kluczy na przedziały jest konfigurowalny.
Domyślnym (i polecanym) algorytmem jest podział ,,losowy'', czyli tak samo jak w Amazon Dynamo na podstawie funkcji mieszającej MD5, dzięki czemu klucze powinny być rozłożone równomiernie pomiędzy węzłami.
Pozostałe dwa dostępne algorytmy nie używają funkcji mieszającej, tylko porównują wartość klucza bezpośrednio z wartościami granicznymi przypisanymi poszczególnym węzłom, dzięki czemu węzły o kolejnych kluczach trafią do tego samego węzła (podobnie jak to ma miejsce w Google BigTable).
Umożliwia to wyszukiwanie zakresów kluczy na podstawie ich wartości.
Niestety to podejście może powodować bardzo nierównomierny rozkład obciążenia w klastrze, a co za tym idzie doprowadzić do niestabilnego działania systemu.
Na szczęście wykorzystując dodatkową rodzinę kolumn (patrz niżej) można własnoręcznie stworzyć indeks pozwalający na dokonywanie tego typu zapytań bez narażania się na ryzyka związanie ze zmianą algorytmu partycjonowania.

\subsection*{Persystencja}

Cassandra, podobnie jak Google BigTable\footnote{patrz strona \pageref{sec:bigtable-architektura-serwera-tabletow}.} zapisuje dane najpierw do logu transakcji, który potem może służyć do ewentualnego odtworzenia stanu bazy w razie awarii, a następnie do struktury danych w pamięci (\emph{memtable}).
Z pamięci co pewien czas dane są zrzucane na dysk tworząc pliki SSTable (Cassandra wykorzystuje tą samą nomenklaturę co Google BigTable).
Ponieważ odczyty często muszą przeglądnąć więcej niż jeden plik SSTable aby znaleźć najnowszą wersję rekordu, konieczne jest łączenie tych plików co pewien czas w większe.

Użytkownik bazy ma do dyspozycji dwie opcje konfiguracji wpływające na trwałość danych.
Możliwe jest albo wykonywanie operacji \verb+fsync()+ przed potwierdzeniem każdej operacji zapisu, albo co pewien określony czas.
W pierwszej z tych konfiguracji zaleca się aby dziennik transakcji znajdował się na innym dysku niż reszta plików, aby uniknąć opóźnienia spowodowanego przesunięciem głowicy przy każdym zapisie.

\subsection*{Wersjonowanie}

Cassandra, w odróżnieniu od Amazon Dynamo i Riak nie dysponuje mechanizmem zegarów wektorowych.
W Cassandrze każda wartość (a właściwie para klucz-nazwa kolumny) ma przypisaną 64-bitową liczbę całkowitą będącą znacznikiem czasowym ostatniej modyfikacji.
W przypadku gdy system wykryje istnienie więcej niż jednej wartości dla danej kolumny, automatycznie jest wybierana najnowsza wartość.

Brak wsparcia dla mechanizmu zegarów wektorowych wydaje się być problematyczny w systemie, który podobnie jak Amazon Dynamo czy Riak należy do rodziny systemów AP w rozumieniu teorii CAP.
Warto jednak zwrócić uwagę, że Dynamo i Riak są systemami typu klucz-wartość i tam wersjonowane są całe rekordy, a zatem jeżeli rekord zostanie zmodyfikowany przez dwóch klientów równocześnie, to nawet jeżeli ci klienci zmodyfikowali całkiem różne ,,pola''\footnote{W systemach typu klucz wartość najczęściej przypisujemy kluczowi wartość będącą zserializowanym obiektem, np. w formacie JSON. Przez pole rekordu rozumiemy tu właśnie pole tego zserializowanego obiektu.} rekordu, konflikt musi zostać rozwiązany przez aplikację.
W przypadku Cassandry każde takie ,,pole'' ma swój znacznik wersji, a co za tym idzie wiele przypadków konfliktów napotykanych w systemach typu klucz-wartość nie ma miejsca w bazie kolumnowej.
W większości pozostałych przypadków można uniknąć powstawania konfliktów odpowiednio modyfikując strukturę danych.

\subsection*{Wyszukiwanie}

\subsubsection*{Model Danych}

Model danych w Apache Cassandra bardzo przypomina ten, który został wcześniej opisany przy okazji Google BigTable\footnote{patrz strona \pageref{google-bigtable-model-danych}.}, ale z pewnymi istotnymi różnicami.
Niestety jedną z tych różnic jest wykorzystanie tej samej nomenklatury, ale w różnych znaczeniach, co może być mylące dla osób które znają model danych bazy Google, dlatego poniżej przedstawiona zostanie lista terminów dotyczących Apache Cassandra wraz z odniesieniami do odpowiadających im terminów dotyczących BigTable.
Bardzo dobre wprowadzenie do modelu danych Apache Cassandra można znaleźć w artykule na blogu Arina Sarkissiana \cite{arin-wtf-is-a-supercolumn}.

\begin{description}
 \item[Keyspace] - dosłownie ,,przestrzeń kluczy''.
 Pozwala na grupowanie rodzin kolumn tworząc pewien rodzaj ,,przestrzeni nazw'' i pozwalając na ustawienie różnych zmiennych konfiguracyjnych.
 Pod wieloma względami przypomina pojęcie bazy, znane z relacyjnych systemów zarządzania bazami danych, takich jak MySQL.
 Nie posiada odpowiednika w BigTable.

 \item[Column Family] - rodzina kolumn.
 Jest to zbiór rzędów, z których każdy posiada klucz, który go identyfikuje oraz dowolną liczbę kolumn.
 Nie istnieje żaden schemat, który wymuszałby aby rzędy tej samej rodziny kolumn miały tą samą strukturę - każdy może mieć zupełnie inny zestaw kolumn.
 Co więcej, kolumny należące do danej rodziny mogą być wyszukiwane po nazwie, stronicowane a ponadto są posortowane przy użyciu komparatora wybranego przez użytkownika.
 Te właściwości sprawiają, że bardzo często dane są przechowywane w nazwach kolumn, a wartości są ignorowane.
 W BigTable odpowiednikiem rodziny kolumn jest tabela.

 \item[Column] - kolumna.
 Jest to trójka (nazwa, wartość, czas ostatniej modyfikacji).
 Nazwa i wartość kolumny nie może przekroczyć rozmiary dwóch gigabajtów.
 W odróżnieniu od Google BigTable (gdzie jest przechowywanych wiele wersji wartości), Cassandra przechowuje tylko najnowszą wersję, a znacznik czasowy jest używany tylko przy rozstrzyganiu konfliktów.

 \item[Super Column] - super kolumna.
 Jest to struktura pozwalająca na grupowanie wielu kolumn.
 W odróżnieniu od rodziny kolumn, super kolumny nie posiadają kluczy.
 Odpowiednikiem super kolumn w BigTable są rodziny kolumn.

 \item[Super Column Family] - rodzina super kolumn.
 Zwykła rodzina kolumn nie pozwala na to aby jej rzędy składały się z kolumn i super kolumn, dopuszczalne są tylko kolumny.
 Rodzina super kolumn z kolei pozwala jedynie aby jej rzędy składały się z super kolumn.
 Stanowi to różnicę w stosunku do BigTable, gdzie rząd tabeli mógł składać się z kolumn zarówno pogrupowanych w rodziny jak i nie.
\end{description}

Do wersji 0.7 Apache Cassandra nie była możliwa zmiana listy rodzin kolumn i przestrzeni kluczy bez ponownego uruchomienia serwera, ale najnowsze wydanie tej bazy wprowadza taką możliwość.

\subsubsection*{Wyszukiwanie rekordów}

Apache Cassandra posiada rozbudowane API pozwalające na filtrowanie i stronicowanie kolumn w ramach pojedynczego rekordu.
Przy wykorzystaniu odpowiedniego mechanizmu partycjonowania, możliwe jest także wyszukiwanie rekordów zakresami po kluczu głównym.

Od wersji 0.7 możliwe jest już tworzenie indeksów drugiego poziomu, co pozwala na wyszukiwanie rekordów po wartości kolumny.
Wcześniej, aby uzyskać taką samą funkcjonalność należało samodzielnie zaimplementować taki indeks poprzez dodanie kolejnej rodziny kolumn, gdzie kluczem byłaby wartość, po której chcemy wyszukać, a nazwami kolumn byłyby identyfikatory wyszukiwanych rekordów.

\subsubsection*{MapReduce}

Cassandra nie posiada własnego, wbudowanego frameworku MapReduce.
Posiada natomiast integrację z systemem Hadoop, dla którego może służyć zarówno jako zbiór danych wejściowych, jak i danych wyjściowych (od wersji 0.7).

\subsection*{Unikalne cechy}

Cassandra łączy w sobie model danych, który jest znacznie bogatszy i łatwiejszy w użyciu niż spotykane w systemach typu klucz-wartość, z technikami znanymi z Amazon Dynamo, które pozwalają na osiągnięcie bardzo wysokiej dostępności systemu.
Dzięki temu, że użytkownikowi jest pozostawiony wybór między konsystencją a dostępnością w obliczu awarii czy podziałów sieci, możliwe jest także stosunkowo proste dopasowanie zachowania systemu do potrzeb aplikacji z niego korzystającej.

Wart uwagi jest fakt, że Apache Cassandra jako jeden z bardzo nielicznych systemów pozwala na taką konfigurację, aby mógł bezproblemowo pracować będąc rozproszonym między wieloma centrami obliczeniowymi, co stawia go w czołówce systemów oferujących najwyższe gwarancje dostępności.

\subsection*{Typowe zastosowania}

Jeszcze zanim ten system został upubliczniony, dowiódł on swojej przydatności w firmie Facebook, gdzie umożliwił wyszukiwanie w miliardach wiadomości które użytkownicy tego serwisu wysyłają do siebie codziennie.
Nie jest zatem przypadkiem, że baza ta została później zaadoptowana przez innych wielkich graczy na rynku aplikacji internetowych: Twitter, Digg i Reddit.
Zainteresowanie jakim Cassandra cieszy się pośród inżynierów największych aplikacji internetowych, świadczy o tym, że system ten należy do grupy najlepiej skalowalnych baz danych dostępnych aktualnie na rynku.

Do tej pory jednak Cassandra nie zdobyła dużej popularności wśród twórców mniejszych aplikacji.
Jest to najprawdopodobniej związane z tym, że tworzenie aplikacji opartych o ten system jest dość trudne i wymaga znacznie większych nakładów pracy niż w przypadku dokumentowych baz danych, czy nawet baz klucz-wartość takich jak Riak, które mogą się pochwalić wbudowanym frameworkiem MapReduce. 
Pomiędzy wersjami 0.6 i 0.7\footnote{W momencie pisania tych słów data wydania wersji 0.7 nie jest jeszcze znana, ale dostępna jest już wersja \emph{Release Candidate} 4.} zostało jednak wprowadzonych wiele zmian (na czele z indeksami drugiego poziomu i możliwością zmiany schematu bez ponownego uruchamiania serwera), które sprawiają, że system ten jest znacznie bardziej przyjazny dla użytkownika niż jeszcze przed rokiem, a zatem jest bardzo prawdopodobne, że nabierze on tym samym większej popularności.

Podobnie jak Riak, Cassandra jest systemem, który powinien dobrze radzić sobie ,,w chmurze''.
Posiada on nawet wbudowane mechanizmy pozwalające na określenie topologii sieci w środowisku Amazon EC2 na potrzeby strategii replikacji.

\subsection*{Przeciwwskazania}

Apache Cassandra jest systemem, w którym odczyty trwają zazwyczaj dłużej niż zapisy, dlatego ten system lepiej się sprawdza w aplikacjach, które wymagają dużej liczby zapisów i modyfikacji.
Znanym ograniczeniem tego systemu jest to, że nie można w nim przechowywać dużych plików, ze względu na to, że protokół Thrift nie udostępnia opcji strumieniowania danych.
Brak transakcji i zegarów wektorowych sprawia także, że problematyczne może być stworzenie w oparciu o ten system aplikacji, w której różni klienci często dokonywaliby zmian tych samych rekordów. 

\subsection*{Dokumentacja i wsparcie}

W internecie jest dostępnych wiele źródeł informacji na temat Apache Cassandra.
Wiele użytecznych informacji znajduje się na wiki projektu, jednak struktura tej strony jest nieprzejrzysta i utrudnia dotarcie do tych danych.

Projekt ten jest aktywnie rozwijany i wyraźnie nabiera co raz większej popularności.

Od dość niedawna możliwe jest też uzyskanie płatnego wsparcia oraz szkoleń dzięki firmie Riptano.

\subsection*{Pomocne odnośniki}

Poniżej zamieszczono kilka odnośników do stron WWW związanych z Apache Cassandra:

\begin{description}
 \item [http://cassandra.apache.org/] - strona domowa projektu
 \item [http://wiki.apache.org/cassandra/] - strona wiki z dokumentacją
 \item [http://www.riptano.com/docs/0.6/index] - alternatywna dokumentacja dla wersji 0.6 oferowana przez firmę Riptano
 \item [http://arin.me/blog/wtf-is-a-supercolumn-cassandra-data-model] - bardzo dobry artykuł opisujący model danych systemu
 \item [http://www.parleys.com/\#st=5\&id=1866] - prezentacja wideo oferująca bardzo dobre wprowadzenie do projektu
\end{description}

\section{HBase}
\label{sec:hbase}

\subsection*{Wstęp}

Apache HBase to system wzorowany na Google BigTable.
Podobnie jak tamten system HBase zależy od dwóch innych systemów: Apache Hadoop i Apache ZooKeeper, pierwszy z których zapewnia replikację (podobnie jak GFS), a drugi przechowuje konfigurację i pozwala na wybór węzła master (tak jak Google Chubby).
HBase powstało początkowo jako rozszerzenie do projektu Hadoop, ale z czasem zostało wydzielone jako osobny projekt, który początkowo przyjął tą samą numerację wersji i harmonogram wydań co projekt-rodzic, ale od najnowszej, jeszcze nie wydanej wersji 0.90, także ta więź zostanie zerwana.

Apache Hadoop, na którym opiera się HBase, to implementacja Google MapReduce w Javie, jednak ponieważ tamten framework zależy od GFS, który nie jest publicznie dostępny, to Hadoop posiada własną implementację rozproszonego systemu plików.

\subsection*{Protokół komunikacji}

Istnieją trzy różne protokoły komunikacji z HBase: Java, Thrift oraz REST (dzięki nakładce o nazwie Stargate).
Najczęściej aktualizowanym i w związku z tym oferującym najwięcej możliwości i najszybciej reagującym na zmiany jest API w Javie, ale pozostałe dwa nie są daleko w tyle.
HBase jest wykorzystywane głównie w połączeniu z aplikacjami pisanymi w języku Java (prawdopodobnie głównie ze względu na ścisłe powiązanie z Apache Hadoop), ale dostępne są też biblioteki dla innych języków programowania.

Podobnie jak większość systemów NoSQL, HBase nie posiada żadnych bardziej zaawansowanych mechanizmów kontroli dostępu.
Jest jednak prawdopodobne, że w przyszłych wersjach taka funkcjonalność zostanie dodana dzięki koprocesorom, czyli funkcjonalności, która ma pojawić się w HBase 0.92 (patrz dalej).
Istnieje już nawet implementacja on nazwie \emph{Secure HBase}, która to umożliwia\footnote{http://hbaseblog.com/2010/10/11/secure-hbase-access-controls/}.

\subsection*{Replikacja}

Podobnie jak Google BigTable, HBase pozostawia replikację rozproszonemu systemowi plików, na którym zapisuje dane.
Teoretycznie, HBase może być wykorzystane z praktycznie dowolnym rozproszonym systemem plików, ale w rzeczywistości system ten jest wykorzystywany tylko w połączeniu z HDFS (\emph{Hadoop Distributed File System}), oraz o wiele rzadziej z KFS (dawniej \emph{Kosmos File System}, teraz \emph{CloudStore}).

\subsubsection*{HDFS}

HDFS jest systemem plików przypominającym swoją architekturą Google File System.
Została ona opisana bliżej w artykule z 2007 roku \cite{hdfs-architecture}, który wprawdzie daje dobry pogląd o tym jak działa ten system, ale jest niestety na tyle stary, że nie jest już autorytatywnym źródłem na ten temat.

Podobnie jak GFS występują w tym systemie dwa typy węzłów: NameNode (jeden w klastrze, odpowiednik węzła Master) oraz DataNode (wszystkie pozostałe, odpowiedniki \emph{chunk-server}).
Tak samo jak GFS, HDFS jest stworzony z myślą przede wszystkim o dużych i bardzo dużych plikach, które podobnie jak w systemie Google są przechowywane podzielone na fragmenty o domyślnej wielkości 64MB, które to fragmenty są także jednostką replikacji.

Największą różnicą między HDFS a GFS jest to, że w systemie Apache awaria węzła NameNode powoduje awarię całego systemu i konieczność jego ponownego uruchomienia.
Jest to o tyle istotne, że węzeł NameNode HDFS pozostaje jedyną ,,piętą Achillesową''\footnote{czyt. \emph{Single Point of Failure}} HBase, które wprawdzie także posiada węzeł Master, ale od niedawna jego awaria już nie powoduje awarii całego systemu.

Inną istotną różnicą między tymi systemami jest to, że ponieważ HDFS powstał na potrzeby narzędzia MapReduce (Hadoop), które z natury nie potrzebuje modyfikować plików po tym jak zostaną one już zapisane, to system ten nie posiada żadnej funkcjonalności przypominającej tą znaną z GFS, która pozwala na dopisywanie ,,rekordów'' na koniec plików przez wielu klientów równocześnie.
Początkowo HDFS był całkowicie pozbawiony możliwości zmieniania zawartości plików, a same pliki były dostępne do odczytu dopiero gdy zostały poprawnie zamknięte\footnote{W związku z tym, HBase jeszcze do niedawna traciło dane w przypadku awarii węzła zapisującego, ponieważ nie dało się ich odczytać z nie zamkniętego poprawnie pliku. W najnowszej wersji (niestabilnej) błąd ten już został naprawiony i da się odzyskiwać dane zapisane w ten sposób.}, później jednak została dodana operacja \verb+append()+, która pozwala na dopisywanie na koniec pliku, ale ponieważ okazało się że nie działa ona całkowicie poprawnie, operacja ta jest dostępna dopiero po ustawieniu odpowiedniej opcji w konfiguracji.

\subsection*{Partycjonowanie}

Apache HBase wykorzystuje ten sam sposób partycjonowania danych co Google BigTable.
Podobnie jak w tamtym systemie, tabele są podzielone na fragmenty o konfigurowalnej wielkości (domyślnie 256MB) nazywane regionami (w BigTable tabletami).
Tak samo jak w BigTable, rzędy są sortowane, ale w odróżnieniu od Apache Cassandra nie jest możliwa zmiana algorytmu sortowania.

Węzły systemu HBase dzielą się na węzeł master oraz serwery regionów (ang. \emph{Region Server}), które są odpowiednikami serwerów tabletów z BigTable.
Każdy serwer regionów (tak jak w Apache Cassandra i BigTable) zapisuje wszystkie zmiany w dzienniku (\verb+HLog+), a następnie odwzorowuje je w pamięci (\verb+MemStore+).
Struktura danych w pamięci jest regularnie zapisywana na dysk do plików (\verb+HFile+), które formatem przypominają pliki SSTable znane z BigTable, a te z kolei co pewien czas są łączone aby uniknąć wyszukiwania w dużej liczbie plików przy odczycie.
Architektura ta jest dokładniej opisana w artykule na blogu Larsa George - jednego z programistów tworzących HBase \cite{george-hbase-storage}, z którego zaczerpnięty został Rysunek \ref{fig:hbase-files}. 

\myfigure{chapters/bazy_kolumnowe/hbase-files.png}{Architektura HBase}{fig:hbase-files}

W starszych wersjach HBase, w przypadku awarii węzła master cały system musiał być uruchomiony ponownie, ale od wersji 0.20 została wprowadzona integracja z odpowiednikiem Google Chubby - Apache ZooKeeper, co pozwoliło na usunięcie tego problemu.

\subsection*{Persystencja}

Za trwałość danych w HBase odpowiada wybrany przez użytkownika system plików.
Jednym z głównych problemów HBase, który został rozwiązany dopiero w nowej, niestabilnej wersji jest wspomniany już problem utraty danych w przypadku awarii serwera regionów, która sprawi, że plik ten nie zostanie poprawnie zamknięty.
Pomijając ten problem, ponieważ każda operacja zmiany wymaga aby dane zostały poprawnie zreplikowane na skonfigurowaną przez użytkownika\footnote{stopień replikacji} liczbę węzłów, to dane zapisane w HBase są zapisywane w sposób trwały w porównaniu z innymi systemami.

\subsection*{Wersjonowanie}

Podobnie jak w Google BigTable, dane są wersjonowane przy pomocy znaczników czasowych (albo innych wartości liczbowych podanych przez użytkownika) i przechowywane są wszystkie wersje wartości, a nie tylko najnowsza jak w Apache Cassandra.

Ponieważ Apache HBase jest systemem CP w rozumieniu Teorii CAP, to przechowywanie wielu wersji każdej wartości ma mniejsze znaczenie niż w systemach typu AP, gdzie konflikty są bardziej prawdopodobne.
Znaczenie to jest tym mniejsze, że HBase posiada możliwość zakładania blokad na rzędy, dzięki czemu możliwe jest zaimplementowanie transakcji w obrębie pojedynczego rekordu (opcjonalnie można też włączyć możliwość blokowania wielu rzędów równocześnie, ale może to się wiązać z negatywnymi skutkami dla wydajności).

\subsection*{Wyszukiwanie}

\subsubsection*{Model Danych}

Model danych HBase jest praktycznie identyczny do modelu danych BigTable\footnote{patrz strona \pageref{google-bigtable-model-danych}.}, oraz w odróżnieniu od Apache Cassandra posługuje się tą samą nomenklaturą nie zmieniając znaczenia poszczególnych terminów.

W HBase jednostką najwyższego poziomu jest \emph{tabela}, która jest posortowanym zbiorem rzędów, z których każdy jest identyfikowany przez klucz, który jest ciągiem bajtów.
Sortowanie odbywa się po kluczu, w porządku leksykalnym, bajtowo.
Każdy rząd może składać się z dowolnej liczby kolumn oraz rodzin kolumn, które z kolei także mogą zawierać kolumny.
Oczywiście tak jak we wszystkich bazach kolumnowych, ,,puste'' wartości nie zajmują miejsca na dysku, a dowolne dwa rzędy mogą się całkowicie różnić zestawem kolumn.

W odróżnieniu od Apache Cassandra nie ma potrzeby deklarowania z góry tabel, mogą one być tworzone dynamicznie, w trakcie działania aplikacji i bez potrzeby restartowania czegokolwiek.

\subsubsection*{Wyszukiwanie rekordów}

Jak przystało na kolumnową bazę, HBase posiada bardzo ograniczone API.
Możliwe jest pobieranie rzędów używając klucza albo zakresu kluczy, a ponadto na poziomie pojedynczego wiersza limitowanie pobranych kolumn (także rodzin kolumn) oraz liczby i zakresu wersji wartości.

Funkcjonalność oferowana przez HBase w zakresie wyszukiwania nie należy do rozbudowanych, stąd też większość ciężaru spoczywa w tym przypadku na użytkowniku, który musi własnoręcznie implementować indeksy umożliwiające znajdowanie rekordów.

\subsection*{Unikalne cechy}

\subsubsection*{Integracja z Apache Hadoop}

Apache Hadoop jest projektem stworzonym z myślą o przetwarzaniu gigantycznych zbiorów danych, a Apache HBase z myślą o przechowywaniu takich zbiorów.
Funkcjonalność oferowana przez ten framework MapReduce stanowi bardzo istotne rozszerzenie dla HBase, ponieważ umożliwia dokonywanie różnego rodzaju obliczeń i transformacji danych, których dane wejściowe i wyjściowe mogą być w niej zapisywane, a co za tym idzie ułatwia tworzenie indeksów, zmaterializowanych widoków, a nawet import i eksport danych.

\subsubsection*{Coprocessor API}

Jedną z ciekawszych funkcjonalności, które będą dostępne w przyszłym wydaniu Apache HBase są koprocesory (ang. \emph{coprocessor}).
Mogą one być porównane do \emph{triggerów} w relacyjnych bazach danych.

Koprocesorem w HBase jest klasa (napisana w Javie), której metody są wywoływane w przestrzeni adresowej serwera w przypadku zajścia różnych zdarzeń.
Do zdarzeń tych należą między innymi zdarzenia związane z tworzeniem i łączeniem plików HFile, oraz zdarzenia, które mają miejsce przy zapisie, odczycie itp. pojedynczych wierszy.
W przyszłości planowane jest także zastosowanie koprocesorów w celu optymalizacji przetwarzania danych na serwerze, w celu zmniejszenia narzutu komunikacyjnego, jaki ma miejsce przy wykorzystaniu Hadoop do wykonywania operacji Map-Reduce.

\subsection*{Typowe zastosowania}

HBase, jako silnie konsystentna baza danych oferuje prostszy model programowania niż na przykład Apache Cassandra, ale z drugiej strony ma także mniejsze możliwości jeżeli chodzi o wyszukiwanie rekordów i filtrowanie kolumn, co sprawia, że nie jest wcale o wiele łatwiejszym systemem w użyciu.
Największą zaletą tej bazy jest bardzo dobra integracja z Apache Hadoop, która czyni ją kuszącą opcją dla aplikacji, które przechowują ogromne ilości danych przetwarzanych na różne sposoby, czyli na przykład wyszukiwarek internetowych.

Do najbardziej znanych aplikacji korzystających z HBase należy StumbleUpon - spersonalizowany silnik rekomendacji różnego rodzaju treści online.
HBase i Hadoop są także wykorzystywane przez wyszukiwarkę internetową firmy Microsoft - Bing\footnote{http://www.theregister.co.uk/2009/05/07/microsoft\_search\_built\_on\_open\_source/}.

\subsection*{Przeciwwskazania}

Apache HBase w swojej obecnej ,,stabilnej'' wersji (0.20.6) nie jest jeszcze bazą wystarczająco stabilną, aby można ją było polecić do zastosowania w środowisku produkcyjnym.
Szczególnie problematyczne jest tu ryzyko utraty danych w przypadku awarii węzła.

Nie wskazane (choć oczywiście możliwe) jest też uruchamianie HBase na platformie Amazon EC2 ze względu na istnienie pojedynczego punktu awarii, a mianowicie węzła NameNode HDFS.

W internecie można znaleźć informacje, że HBase jest zbyt wolne aby nadawać się dla aplikacji on-line\footnote{http://www.metabrew.com/article/anti-rdbms-a-list-of-distributed-key-value-stores/}, jednak przykład StumbleUpon pokazuje, że nie jest to prawdą.

\subsection*{Dokumentacja i wsparcie}

HBase, w odróżnieniu od większości systemów NoSQL nie jest tworzone głównie przez programistów jednej firmy, ale przez osoby należące do wielu różnych instytucji.
Zaletą takiego układu jest to, że łatwiej jest uzyskać darmową pomoc od ludzi, którzy pracują nad tą aplikacją ponieważ ich firmy nie zarabiają na sprzedawaniu tych usług.
Wadą jest jednak to, że brakuje firmy, która oferowałaby wyspecjalizowane usługi, szkolenia i gwarantowane wsparcie dla tego produktu.

Od września 2009 HBase zostało włączone do pakietu produktów opartych o Apache Hadoop oferowanego przez firmę Cloudera, co przynajmniej częściowo mityguje tą wadę.

Dokumentacja projektu, podobnie jak w przypadku Apache Cassandra jest dość chaotyczna i dostępna jedynie w postaci wiki.
W internecie jest dostępnych stosunkowo niewiele wideo-prezentacji opisujących tą bazę, ale za to można znaleźć dość dużo informacji o tym systemie na różnych blogach.

\subsection*{Pomocne odnośniki}

Poniżej zamieszczono kilka odnośników do stron WWW związanych z Apache HBase:

\begin{description}
 \item [http://hbase.apache.org/] - strona domowa projektu
 \item [http://wiki.apache.org/hadoop/Hbase] - strona wiki z dokumentacją
 \item [http://www.parleys.com/parleysserver/indexing/presentation.form?id=1859] - prezentacja wideo oferująca bardzo dobre wprowadzenie do projektu
 \item [http://nosqltapes.com/video/ryan-rawson-on-hbase-at-stumbleupon] - wywiad z jednym z twórców bazy, który opowiada między innymi o zastosowaniach HBase w firmie StumbleUpon
\end{description}

\chapter{Bazy Dokumentowe}

\section*{Streszczenie}
W tym rozdziale opisane zostaną dwie bazy dokumentowe: CouchDB i MongoDB.

\section{CouchDB}
\label{sec:couchdb}

\section{MongoDB}
\label{sec:mongodb}
\chapter{Bazy Grafowe}

\section*{Streszczenie}
W tym rozdziale opisane zostaną dwie bazy grafowe: Neo4J i Infinite Graph.

\section{Neo4J}
\label{sec:neo4j}

\subsection*{Wstęp} 

\subsection*{Protokół komunikacji}

\subsection*{Replikacja}

\subsection*{Partycjonowanie}

\subsection*{Persystencja}

\subsection*{Wersjonowanie}

\subsection*{Wyszukiwanie}

\subsection*{Unikalne cechy}

\subsection*{Typowe zastosowania}

\subsection*{Przeciwwskazania}

\subsection*{Dokumentacja i wsparcie}

\subsection*{Pomocne odnośniki}

\section{Infinite Graph}
\label{sec:infinite_graph}

\subsection*{Wstęp} 

\subsection*{Protokół komunikacji}

\subsection*{Replikacja}

\subsection*{Partycjonowanie}

\subsection*{Persystencja}

\subsection*{Wersjonowanie}

\subsection*{Wyszukiwanie}

\subsection*{Unikalne cechy}

\subsection*{Typowe zastosowania}

\subsection*{Przeciwwskazania}

\subsection*{Dokumentacja i wsparcie}

\subsection*{Pomocne odnośniki}

\chapter{Skalowalne Bazy Relacyjne}

\section*{Streszczenie}
W tym rozdziale opisana zostanie skalowalna bazy relacyjna: VoltDB.

\section{VoltDB}
\label{sec:voltdb}
\chapter{Wnioski}

\section*{Streszczenie}
Praca zostanie zakończona prezentacją wniosków wynikających ze zdobytych informacji na temat systemów NoSQL oraz przeprowadzonych eksperymentów.
Dołączona bibliografia zawierać będzie notatki dotyczące poszczególnych pozycji, a w załączniku do pracy, dla wygody przyszłych czytelników, dołączę komplet artykułów wymienionych w bibliografii.

\chapter{Zakończenie}

\section*{Streszczenie}
Kilka słów podsumowania.

\bibliography{bibliografia}

%-------------------------------------------------------------------------------
% Załączniki
%-------------------------------------------------------------------------------

\printindex
\listoffigures
\listoftables

% na potrzeby pisania pracy
\newpage
\listoftodos

\end{document}
