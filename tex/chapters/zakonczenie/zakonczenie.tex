\chapter{Zakończenie}

Celem niniejszej pracy było przybliżenie czytelnikowi najbardziej znanych, nierelacyjnych systemów bazodanowych.
Zcharakteryzowane zostało siedem systemów, z których każdy znacznie różni się od wszystkich pozostałych.
Przedstawione zostały także teorie i wzorce, które wpłynęły na architekturę tych systemów i dokonano porównania tego jak opisane systemy realizują te wzorce, jakie mają cechy wspólne i czym się różnią.

Dostępność nowych, nierelacyjnych systemów baz danych, zebranych pod wspólną nazwą ,,Systemów NoSQL'', zachęca twórców aplikacji do porzucania sprawdzonych przez bardzo wiele lat rozwiązań, którymi są relacyjne bazy danych, na rzecz nowych systemów bazodanowych, które kuszą wizją ,,skalowalności'', ukrywając równocześnie trudności i problemy jakie wiążą się z ich zastosowaniem.

Podjęcie wyboru między relacyjną bazą danych a jednym z systemów NoSQL, a szczególnie wybór jednego z wielu systemów nierelacyjnych jest wyborem bardzo trudnym i wymagającym dobrego zrozumienia tematyki rozproszonych systemów bazodanowych i wymagań biznesowych aplikacji na rzecz której ten wybór jest dokonywany.
Poniżej zamieszczone zostały przykładowe pytania jakie powinien sobie w takiej sytuacji zadać architekt systemu. 

\section*{Przykładowe kryteria wyboru}

\begin{itemize}
 
 \item Jaka będzie maksymalna ilość danych przechowywanych w systemie?

 Niektóre systemy, takie jak Riak, HBase i Apache Cassandra, umożliwiają przechowywanie ogromnych ilości danych, ale wiąże się to z utrudnieniami związanymi z ograniczeniami w wyszukiwaniu i modelowaniu danych.
 Z kolei inne systemy (Redis, Neo4J) oferują znacznie bogatszy model danych, ale kosztem ograniczonej skalowalności.

 Częstym powodem porzucania relacyjnych baz danych na rzecz systemów NoSQL jest to, że zmiana schematu dużej bazy danych może trwać wiele godzin a czasem nawet dni, co ogranicza możliwości rozwoju aplikacji korzystającej z takiej bazy.
 Jeżeli jednak dla rozpatrywanej aplikacji czas takich migracji byłby liczony tylko w minutach, warto się dobrze zastanowić, czy ryzyko związane z wyborem nietypowego rozwiązania jest warte potencjalnych zysków.

 \item Jakiego rodzaju koszty wiążą się z niedostępnością aplikacji?

 Dla niektórych aplikacji, każda minuta niedostępności może wiązać się z poważnymi konsekwencjami finansowymi.
 Do systemów takich należą na przykład systemy wspierające produkcję czy systemy płatności internetowych.

 W innych przypadkach czasowe ograniczenie dostępności systemu nie wiąże się z wielkimi kosztami, dlatego bardziej opłaca się minimalizować koszt wytworzenia aplikacji, która byłaby bardziej skomplikowana (a zatem droższa), jeżeli wykorzystywałaby system stawiający na wysoką dostępność (Riak, Cassandra, CouchDB).

 \item Czy aplikacja potrzebuje specyficznych funkcjonalności zapewnianych przez system?

 Niektóre z omawianych w tej pracy systemów oferują unikalne możliwości dla twórców aplikacji.
 Na przykład CouchDB pozwala na tworzenie aplikacji, gdzie każdy klient posiada własną kopię (części) bazy danych, co pozwala mu pracować z aplikacją nawet przy braku połączenia z centralnym serwerem.
 Redis z kolei posiada implementacje wielu struktur danych, takich jak zbiory, listy i tablice mieszające.
 Jeszcze bardziej unikalny zestaw funkcjonalności posiada Neo4J, która to baza oferuje niespotykane nigdzie indziej możliwości modelowania domeny aplikacji (graf jest bardzo naturalnym sposobem odwzorowania rzeczywistości), a także umożliwia bardzo efektywną realizację algorytmów grafowych.

 Specyficzne potrzeby aplikacji są najlepszym powodem aby wykorzystać bazę danych najbardziej przystającą do wymagań, ale ponieważ każda taka funkcjonalność została dodana pewnym kosztem, to warto się upewnić, że nie ponosi się kosztów zbędnych funkcjonalności systemu.

\end{itemize}

Kryteriów takich jak te powyżej jest bardzo wiele: 
Jak bardzo liczy się opóźnienie? 
Jaki jest stosunek zapisów do odczytów? 
Czy aplikacja będzie skalowana dynamicznie, czy węzły będą dodawane i usuwane z klastra bardzo rzadko?

Odpowiedź na nie musi być istotną częścią procesu decyzyjnego -- zbyt wiele zespołów tworzących aplikacje internetowe ślepo podążając za modą wybiera jako główne narzędzie persystencji systemy NoSQL takie jak MongoDB, będąc całkowicie pozbawionymi świadomości wad tych systemów i możliwych alternatyw.
Celem autora było przedstawienie tych alternatyw, ich wad i zalet, oraz informacji, które pozwolą czytelnikowi na łatwiejsze zrozumienie wad i zalet innych systemów, dla których zabrakło tu miejsca.