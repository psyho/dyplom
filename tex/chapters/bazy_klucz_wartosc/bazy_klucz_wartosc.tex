\chapter{Bazy Klucz-Wartość}

\section*{Streszczenie}
W tym rozdziale opisane zostaną dwie bazy klucz-wartość: Redis i Riak.
\todo{Rozszerzyć wstęp}

\section{Redis}
\label{sec:redis}

\subsection*{Wstęp} 

Redis jest jedną z najbardziej popularnych baz NoSQL.
Swoją popularność zawdzięcza on niezwykle bogatemu jak na bazę typu klucz-wartość API, oraz bardzo dużej wydajności.
Bardzo często system ten jest wykorzystywany w aplikacjach internetowych równolegle z relacyjnymi bazami danych, najczęściej jako kolejka zadań albo w zastępstwie Memcached jako cache oraz baza, w której przechowywana jest sesja użytkownika.

Mimo, że jest on klasyfikowany jako baza typu klucz-wartość, Redis nie jest horyzontalnie skalowalny.
Nie jest to zazwyczaj bardzo dużym problemem, ponieważ Redis jest w większości przypadków wystarczająco wydajny, aby jeden węzeł był w stanie obsłużyć wszystkie zapytania generowane przez aplikację.

\subsection*{Protokół komunikacji}

Redis używa bardzo prostego, tekstowego protokołu komunikacji.
Istnieje możliwość autentykacji, ale tylko przy pomocy hasła zapisanego w konfiguracji serwera.
Nie ma możliwości tworzenia wielu użytkowników i określania ich uprawnień.
Ponieważ hasło jest przesyłane otwartym tekstem, a połączenia nie są w żaden sposób szyfrowane, Redis nie należy do najbezpieczniejszych rozwiązań na rynku.

Dzięki prostemu protokołowi, biblioteki do komunikacji z tym systemem są dostępne praktycznie dla każdego języka programowania.
Dla niektórych języków programowania dostępne są także biblioteki wyższego poziomu, które pełnią rolę bardzo podobną do bibliotek ORM (Object-Relational Mapping).
Biblioteki takie jak Ohm dla języka Ruby pozwalają na mapowanie obiektów na strukturę kluczy w Redis, włącznie z indeksowaniem wartości niektórych pól, co umożliwia wyszukiwanie rekordów po czym innym niż wartość klucza. 

\subsection*{Replikacja}

Redis posiada mechanizm replikacji w trybie master-slave.
Replikacja jest asynchroniczna, co oznacza, że dane odczytane z serwerów slave mogą być czasem nieaktualne.
Często Redis jest konfigurowany w taki sposób, aby serwer master w ogóle nie dokonywał zapisów na dysk twardy, tylko pozostawiał tą rolę serwerowi lub serwerom slave.
Rozwiązanie takie pozwala na zwiększenie wydajności.

\subsection*{Partycjonowanie}

Redis nie obsługuje partycjonowania w wersji 2.0.
Niektóre biblioteki pozwalają wprawdzie na implementację partycjonowania po stronie klienta, ale są to bardzo prymitywne rozwiązania, które nie mogą w żaden sposób konkurować z partycjonowaniem po stronie serwera, jakie jest zaimplementowane np. w omawianym w kolejnym rozdziale systemie Riak.

\subsubsection*{Redis Cluster}

W planach jest implementacja tak zwanego \emph{Redis Cluster}, który pozwoli na horyzontalne skalowanie systemu poprzez replikację i partycjonowanie danych.
Redis Cluster będzie systemem typu CA (Consistent-Available).
W systemie będą występować cztery rodzaje węzłów:

\begin{description}
 \item[Data Node] (węzeł danych) - węzły przechowujące dane.
 Skalowanie systemu odbywa się poprzez zwiększanie liczby tych węzłów.

 \item[Configuration Node] (węzeł konfiguracyjny) - węzeł przechowujący metadane o całym systemie, takie jak listy \emph{Data Node} i \emph{Proxy Node}.
 W razie awarii tego węzła, system wprawdzie nie zaprzestaje normalnej pracy, ale nie jest w stanie obsłużyć sytuacji awaryjnych takich jak awaria innego węzła.
 Ten węzeł tylko przechowuje dane konfiguracyjne, węzłem odpowiedzialnym za ich obsługę jest Handling Node.

 \item[Proxy Node] (węzeł proxy) - węzły odpowiedzialne za koordynację zapytań w systemie.
 Zapytania w systemie są zawsze kierowane do węzłów Proxy, które przekazują je następnie do jednego (w przypadku odczytów) lub wszystkich (w przypadku operacji zmieniających dane) węzłów, na których odpowiednie rekordy są zapisane.
 Awarie są wykrywane przez Proxy Node i informacja o nich jest zapisywana w węźle konfiguracyjnym.

 \item[Handling Node] (węzeł zarządzający) - to klient, który obsługuje dane zapisane w węźle konfiguracyjnym.
 Zajmuje się on zmianą przydziału zakresów kluczy do węzłów w razie dodania lub usunięcia węzła, lub jego awarii.
 Najczęściej Handling Node i Configuration Node są umieszczone na tym samym fizycznym węźle.
\end{description}

\subsection*{Persystencja}

Persystencja

\subsection*{Wersjonowanie}

Wersjonowanie

\subsection*{Wyszukiwanie}

Wyszukiwanie

\subsection*{Unikalne cechy}

Unikalne cechy

\subsection*{Typowe zastosowania}

Typowe zastosowania

\subsection*{Przeciwwskazania}

Przeciwwskazania

\subsection*{Dokumentacja i wsparcie}

Dokumentacja i wsparcie

\subsection*{Pomocne odnośniki}
 
Pomocne odnośniki

\section{Riak}
\label{sec:riak}

\subsection*{Wstęp} 

Wstęp

\subsection*{Protokół komunikacji}

Protokół komunikacji

\subsection*{Replikacja}

Replikacja

\subsection*{Partycjonowanie}

Partycjonowanie

\subsection*{Persystencja}

Persystencja

\subsection*{Wersjonowanie}

Wersjonowanie

\subsection*{Wyszukiwanie}

Wyszukiwanie

\subsection*{Unikalne cechy}

Unikalne cechy

\subsection*{Typowe zastosowania}

Typowe zastosowania

\subsection*{Przeciwwskazania}

Przeciwwskazania

\subsection*{Dokumentacja i wsparcie}

Dokumentacja i wsparcie

\subsection*{Pomocne odnośniki}
 
Pomocne odnośniki