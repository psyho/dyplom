\chapter{Bazy Dokumentowe}

\section*{Streszczenie}
W tym rozdziale opisane zostaną dwie bazy dokumentowe: CouchDB i MongoDB.

\section{CouchDB}
\label{sec:couchdb}

\subsection*{Wstęp}

Apache CouchDB to dokumentowa baza danych oferująca szereg nie spotykanych w innych systemach funkcjonalności, takich jak framework MapReduce, który pozwala na tworzenie indeksów drugiego poziomu czy bardzo rozbudowaną funkcjonalność replikacji, która pozwala na tworzenie systemów, w których klient może przechowywać lokalnie jedynie ten fragment bazy, który go interesuje (umożliwiając tym samym pracę nawet w warunkach utraty połączenia z siecią) i pozostawić funkcjonalność synchronizacji danych CouchDB.

System ten jest jednym z najdłużej dostępnych na rynku - baza ta została udostępniona już w 2005 roku.
O dojrzałości i popularności tego projektu świadczy ogromna ilość dostępnej dokumentacji oraz aktywna społeczność.

Apache CouchDB jest systemem napisanym w języku Erlang i dostępnym na licencji open source (Apache License 2.0).

\subsection*{Protokół komunikacji}

CouchDB dysponuje tylko jednym interfejsem komunikacji z bazą: REST po HTTP.
Z jednej strony tego typu interfejs jest zaletą, gdyż HTTP jest prostym protokołem, obsługiwanym z łatwością w każdym praktycznie języku programowania, a także zastosowanie tego protokołu umożliwia uzupełnienie CouchDB poprzez zastosowanie serwerów proxy czy load balancerów, których dla tego protokołu jest bardzo wiele.
Zastosowanie tego protokołu umożliwia wręcz tworzenie aplikacji internetowych, które nie wymagają serwera aplikacyjnego - wystarczy klient w języku JavaScript wykonywany w przeglądarce i baza.
Wadą wykorzystania HTTP jako protokołu komunikacji jest to, że w porównaniu z binarnymi protokołami transmisji jest dość wolny, gdyż wymaga większej przepustowości i więcej czasu na serializację i deserializację danych.

Dla CouchDB dostępne są biblioteki dla praktycznie wszystkich popularnych języków programowania, jak również biblioteki wysokiego poziomu.
Istnieją także implementacje CouchDB w innych językach niż Erlang, dzięki czemu CouchDB może być instalowane na urządzeniach mobilnych a nawet, wykorzystując mechanizmy HTML5, w przeglądarce internetowej.

\subsection*{Replikacja}

\subsection*{Partycjonowanie}

\subsection*{Persystencja}

\subsection*{Wersjonowanie}

\subsection*{Wyszukiwanie}

\subsection*{Unikalne cechy}

\subsection*{Typowe zastosowania}

\subsection*{Przeciwwskazania}

\subsection*{Dokumentacja i wsparcie}

\subsection*{Pomocne odnośniki}

\section{MongoDB}
\label{sec:mongodb}

\subsection*{Wstęp} 

\subsection*{Protokół komunikacji}

\subsection*{Replikacja}

\subsection*{Partycjonowanie}

\subsection*{Persystencja}

\subsection*{Wersjonowanie}

\subsection*{Wyszukiwanie}

\subsection*{Unikalne cechy}

\subsection*{Typowe zastosowania}

\subsection*{Przeciwwskazania}

\subsection*{Dokumentacja i wsparcie}

\subsection*{Pomocne odnośniki}
